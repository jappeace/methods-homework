\documentclass{article}
\usepackage{amsmath}
\usepackage{txfonts}
\usepackage{booktabs}
\usepackage{color}
\usepackage{bussproofs}
\usepackage{graphicx}
\usepackage{pifont}
\usepackage{qtree}
\usepackage{tikz}
\usetikzlibrary{automata,arrows,positioning,calc}
\usepackage{listings}
\usepackage{hyperref}
\newenvironment{scprooftree}[1]%
{\gdef\scalefactor{#1}\begin{center}\proofSkipAmount \leavevmode}%
{\scalebox{\scalefactor}{\DisplayProof}\proofSkipAmount \end{center} }

\begin{document}
\lstset{language=Java}
\author{Jappie Klooster}
\title{Thoughts about cognition}
\maketitle

\section{Introduction}
This will be an introduction to cognitive proccessing
coming from someone who followed
` hbo informatica' as pre study.

In here I will write down the things I think are important.

\section{Research and design}
This chapter will talk about how to set up experiments and how to draw
conclusions from them. It will also discuss what tools are available.
Matlab was also discussed, but considering my background I will not spend
time on this subject.

\subsection{Brains}
The human and monkey brain are rather similiar (in shape mostly).

\subsubsection{Why do we have brains?}
Reproduction is an important factor, there exists animals that consume their
own brain ones they start making their offspring.

\subsection{Common denominator}
All the brains have neurons.

Incoming singals arrive at denrites, and are integrated to trigger an outgoing.
signal (or not) at the axon hillock.
Cell firing action potential is sent down the axon, which is often insulated
with meylin to increase conduction peed, jumping between the nodes of ranvier.

When the action potential arrives at the terminal bouton, neurotransmitters
are realeased to activate the next cell.

\subsubsection{Action potentials}
At the baseline, neuron potential difference between the outside and inside
of the cell is about -70mv. Incoming singals(arrows) cause changes that
make the inside of teh cell more positive. When a treshold is reached a
cascade of events is triggeredthat make the inside of the cell very positive,
very quickly. This discharge is the action potential and starts at the hillock.
It travels trough the axon to the next cell.

So the charge level slowly built until a treshold, and then releases. This
causes chemicals to jump at the terminals. Which is treated as an incoming
singal to the next neuron.

\subsubsection{Neuro transmision}
Neuro transmission is done with neurotranmitters, there are two main
transmitters, Glu, which makes the next cell more likely to fire, its
exictatory. GABA makes the next cell less likely to fire, its inhibitory.

Neurotransmitters (NT) are stored in vesicles. Electrical signal causes
vesicles to bind with the membrane of the terminal and thus release NT
out of the cell into synaptic cleft. NT spread and ome bind to next
cell at particluar locations: receptors. Binding to receptor causes
changes in the cell that may result in it signaling/firing itself.

The synaptic cleft is the gap between the axon terminal and
the dendritic spine.

\paragraph{Receptors} Usually a protein channel. NT can bind to it
and change its shape. When open: Ions can flow trough it and
chagne the electrical potential of the cell (try to excite or
inhibit it).

Many receptors come in the form of `channels'. This is a chanal,
aka potential opening, between the in and outside of the cell.
Depending on how it's folded it may allow ions to pass or not.
Its shape also makes it selecive: Only some ions can pass.

There are many different kinds (Glu, Gaba, dopamine etc).
Sensory receptors repsond to the outside world instead of signals
from other neurons: Light, pressure, taste or smell molucles vibration etc.

Dopamine and serotine are very common. Not main transmitter but important
modulators of activity. Not all recptors are channals. We also have sensory
recptors. They're special and can be very different. Complex molecules bind
to chemical recptors in the nose and on the tongue. Hair cels in the cockle
(ear) respond to vibration of a particular frequency.

\subsection{Sensory systems}
Converting energy from outside the body to neural signals trough: stimulus.
Modification of signal, transduction or transfer.

Flavour is a multi sensoy phonomon. Olfaction and ustation strong connections
to limbic system. Chemo signalling affects behoviour, perception and emmotions.

Chemical senses appear to have a strong link to social behaviour.

\subsubsection{The pupil}
Allows more or less light to enter the eye. More light usually not a good idea.
Dilation: arousal, Contrsiction: Relaxation.

Sensory aquisition \& rejection: emotrion affects pupil respones.

\subsubsection{Smell}

MHC: Major histocompatibility complex. Genes with product that is part of
immume systems Having more different. Smelling mice and vice versa.

Humans can small differences beteween mouse families only differing in MHC.
mice can recognise human MH-types from urine.


\paragraph{Do we use it?} Contraceptives seem to interfere with natural mate
selection: A big deal because similiar MHC tends to reduce succesfull
pregnacies.

\subsubsection{Receptive fields}
Description depends on modality. Somatosensory, Auditory, Visual, Gustation.

Pain for example spreads from 1 point to the entire surrounding area, even
if its not affected.

\section{Langauge}
Trying to figure out what langauge is.

\section{Langauge and Context}
In here we will focus on trying to reduce the influence of context or
at least try to explain it.

\end{document}
