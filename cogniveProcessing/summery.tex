\documentclass{article}
\usepackage{amsmath}
\usepackage{txfonts}
\usepackage{booktabs}
\usepackage{color}
\usepackage{bussproofs}
\usepackage{graphicx}
\usepackage{pifont}
\usepackage{qtree}
\usepackage{tikz}
\usetikzlibrary{automata,arrows,positioning,calc}
\usepackage{listings}
\usepackage{hyperref}
\newenvironment{scprooftree}[1]%
{\gdef\scalefactor{#1}\begin{center}\proofSkipAmount \leavevmode}%
{\scalebox{\scalefactor}{\DisplayProof}\proofSkipAmount \end{center} }

\begin{document}
\lstset{language=Java}
\author{Jappie Klooster}
\title{Thoughts about cognition}
\maketitle

\section{Introduction}
This will be an introduction to cognitive proccessing
coming from someone who followed
` hbo informatica' as pre study.

In here I will write down the things I think are important.

\section{Research and design}
This chapter will talk about how to set up experiments and how to draw
conclusions from them. It will also discuss what tools are available.
Matlab was also discussed, but considering my background I will not spend
time on this subject.

\subsection{Brains}
The human and monkey brain are rather similiar (in shape mostly).

\subsubsection{Why do we have brains?}
Reproduction is an important factor, there exists animals that consume their
own brain ones they start making their offspring.

\subsection{Common denominator}
The soma or ``cell body'' is the bulbous end of a
neuron, containing the cell nucleus.

Dendrites are the branched projections of a neuron that act to
propagate the electrochemical stimulation received from other neural
cells to the cell body, or soma, of the neuron from which the
dendrites project

\end{document}
