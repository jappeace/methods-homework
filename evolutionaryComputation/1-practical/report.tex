\documentclass{article}
\usepackage{amsmath}
\usepackage{txfonts}
\usepackage{booktabs}
\usepackage{color}
\usepackage{bussproofs}
\usepackage[pdftex]{graphicx}
\DeclareGraphicsRule{*}{mps}{*}{}
\usepackage{pifont}
\usepackage{qtree}
\usepackage{emp}
\usepackage{listings}
\usepackage{hyperref}
\usepackage{svg}
\newenvironment{scprooftree}[1]%
{\gdef\scalefactor{#1}\begin{center}\proofSkipAmount \leavevmode}%
{\scalebox{\scalefactor}{\DisplayProof}\proofSkipAmount \end{center} }


\newcommand{\brcell}[2][l]{%
	\begin{tabular}[#1]{@{}l@{}}#2\end{tabular}}
\begin{document}
\begin{empfile}
\begin{empcmds}
input metauml;
\end{empcmds}
\author{Jappie Klooster}
\title{A practical look at genetic algorithms}
\maketitle

\section{Introduction}
This is a report of the evolutionary practical assignment 1.
In here I will discuss the execution of the assignment, what it was, which
methodology was used to execute the assignment.

The intended audience is people familiar with programming.

\section{Assignment}
The assignment was to implement the following functions and algorithms and
then do experimentation with help of these.

\subsection{Functions}
\subsubsection{Counting functions}
\[x_i \in \{0,1\}:CO(x_1\dots x_l)=\sum^l_{i=1}x_i\]
\[x_i \in \{0,1\}:CO(x_1\dots x_l)=\sum^l_{i=1}i x_i\]
\subsubsection{Trap function}
\[TF(x_1\dots x_l) = \sum^{\frac{l}{k}-1}_{j=0}B(x_{jk_1}\dots x_{jk+k})\]

with:

\[B(x_1\dots x_k)=\left\{
	\begin{matrix}
		k & \mbox{ if } &   CO(x_1\dots x_k) = k \\
		k - d - \frac{k-d}{k-1} CO(x) & \mbox{ if } & CO(x_1\dots x_k) < k \\
	\end{matrix}
\right. \]

\subsection{Experiment}
After implementing these functions we were tasked to make an experiment of it
and make some graphs of the results and make a conclusion based on the results.

\section{Methodology}

\subsection{Language choice}
I chose to use Scala to implement this project. The main reason being that I've
played with Scala before but never have done anything constructive with it.

Scala is both a functional and an object orientated program that runs on the JVM
It compiles to byte-code and is statically typed.

It also can use any Java library and integrates seamlessly with Java, (although
calling methods the methods of a Scala project from a Java project can be
a little clumsy sometimes, because Scala supports operator overloading in
a strange way, but its possible).

Other options I considered were:
\begin{itemize}
	\item Java, Java is an industry standard language, but the syntax of
		Java is very verbose. So when I have the choice I try to avoid it.
		I've also used a lot of Java when I was younger so I won't learn
		much by using it.
	\item C\#, the language is good, a lot less verbose then Java
		but setting it up is difficult since I use Linux.
		It also prefers object orientated rather than
		functional programming. Scala's type system is also more advanced than
		C\#'s, so its in a way more object orientated.
	\item Rust/Go, I'm curious about both these languages, but I'm not at
		all familiar with them, so I probably should do first a little
		hobby project with them before starting doing something serious like
		this, this is in contrast to Scala, with which I already played before.
	\item Python, I think the project is in that grey area where weakly
		typed languages become less powerful in use. Especially considering
		the fact how quickly these algorithms blow up in runtime. Static type
		checking then helps solving a big part of the bugs before execution
		and since execution will be slow, this saves a lot of time.
	\item C++, C++ is extremely fast in execution times. However implementation
		time is quite long compared to almost everything else.
		Errors are for example reported poorly and
		the syntax is inconsistent. Since execution speed isn't a top priority
		I decided to not use C++.
\end{itemize}

\subsection{Architecture}
The code is split up into 4 parts, first there are the case classes. These are
immutable classes that hold the domain information (Java beans).
Then there are the algorithm classes and objects. These are pretty straight
forward logical structures. They implement the atoms of possible GA's, such
as for example uniform crossover. After that comes the experiment and the result
classes, the experiment just uses the GA in a specific way.
Finally there is the presentation object \emph{Main}. The only thing it does
is interpret the experiment into a latex structure. There is no
diagram discussing Main, because there is little architecture involved.

\subsubsection{Case classes}
The case classes add meaning to the structures.
A case class is a Scala concept were you just define the attributes. Then
Scala generates the `equals', `hashCode' and `toString' methods for you.
They're also by default immutable, which is something I like using. You
can make mutations by using the generated copy function, in which you can
specify what is changed to create a new copy with the change.

\begin{figure}[ht!]
\centering
\begin{emp}[classdiag](20,20)
Class.mem("Member")("+Fitness:Int", "+genes:String")();
Class.pop("Population")("+members:Seq[Member]", "+factory:String=>Member")();
Class.pairedPop("PairedPopulation")("+members:Seq[Pair]","+factory:String=>Member")(
"+pairedToPop(how):Population");
Class.pai("Pair")("+father:Member","+mother:Member")();

Instance.population("Population")(
"createOnesAndZeros(factory:String=>Member,popCount:Int,geneLength:Int):Population"
);

leftToRight.top(20)(pop,pairedPop);
topToBottom.left(20)(population, pop);
topToBottom.left(60)(pop,mem);
leftToRight.top(80)(mem, pai);

drawObjects(mem,pop,pairedPop,pai,population);
link(association)(pai.n -- pairedPop.s);
item(iAssoc)("*")(obj.se = pai.n);
link(association)(mem.e -- pai.w);
item(iAssoc)("2")(obj.sw = mem.e);

link(association)(mem.n -- pop.s);
item(iAssoc)("*")(obj.se = mem.n);
\end{emp}
\caption{Case classes}
\end{figure}

\paragraph{Member} A \emph{Member} is something with a gene. The fitness value gets pre calculated by
the factory, so that every members' fitness is only calculated once. This can
be done reliably because the member is immutable. There is no way to modify
the genes so the fintness value does not have to change.

\paragraph{Population} The members are generated in batches and stored into a
\emph{Population}, either by the paired
population or by the Population companion object which provides a
`createOnesAndZeros' function.
A population will also have a function with it to generate new members as
data in this case, it drags the specified function along, its like a
function pointer.
The population class doesn't use it itself, but when its members are paired
up into \emph{Pair}s. A \emph{PairedPopulation} can use the creation function
of the original population to create new members.

\paragraph{PairedPopulation}
The \emph{PairedPopulation} has also a utility function pairedToPop,
this receives a how, in which it can
insert zipped genes (char by char paired up strings) and from which comes
a char sequence. This how is implemented by one of the crossing methods.
This abstraction was added because the code around the cross function was
always the same.
The resulting string from `how' is then inserted into a member
factory to create a new member (the member factory then allows mutation
or obfuscation of the new genes).

I've thought about merging PairedPopulation and population into
one class. However doing that will probably not make things more obvious.
Although it does makes sense in that the population would actually use
the function it drags along.

\newpage
\subsubsection{Algorithm}
The objects defined above all have functions with the types
same named evolution attributes. So the functions are just basically
options you can use to create an
evolution from. There are some exceptions which allow wrapping is the
withCoinFlipMutation function in
memberFactories. This function acts as a decorating function. So you can wrap
it around another function to create a new function.

Objects are just singletons, or Java static space. I use the object keyword in
Scala instead of class So I chose also to use an object in the UML
representation. Scala seems to stick pretty close to UML notation anyways.
The objects don't have types however, because they are the types. They're
language enforced singletons.

\begin{figure}[ht!]
\centering
\begin{emp}[classdiag](20,20)
Class.evo("Evolution")(
"+evaluation: FitnessEvaluator",
"mateSelector:IMateSelector",
"offSpringGenerator: PairedPopulation => Population",
"fittestFilter: (Population, Population) => Population",
"hasGoodEnoughSolution: Population => Boolean")(
"startGenetic(seed:Population): Seq[Population]",
"genetic(prev:Seq[Population]): Seq[Population]",
"step(parents:Population): Population",
);
Instance.fit("Evaluation")(
	"uniformlyScaled()",
	"linearlyScaled()",
	"blockValuation(block:Seq[Float])",
	"createProbe(function):FitnessEvaluator"
);
Instance.filt("FittestFilter")(
	"killParents()",
	"truncateElitism()",
	"tournementElitism()"
	);
Instance.select("MateSelection")(
	"tournementSelect()"
);
Instance.factor("MemberFactories")(
	"tightlyLinkded()",
	"randomlyLinked()",
	"withCoinflipMutation(function):function",
);
Instance.off("OffspringGenerator")(
	"uniformCross()",
	"singlePointCross()",
	"twoPointCross()"
);

topToBottom.left(20)(evo,fit,filt,factor);
leftToRight.top(20)(fit,off);
leftToRight.top(20)(filt,select);

drawObjects(evo,fit,factor,filt,select,off);
\end{emp}
\caption{Algorithm}
\end{figure}

\paragraph{Evolution}The evolution class is the common structure of a GA(Gentic algorithm){.} 
So creating an instance of Evolution is like creating a new GA{.}

To construct an Evolution you need to pass a \emph{valuation function},
a \emph{mateselector}, An \emph{offspring generator},
a \emph{fittestFilter}. And a function that determines if there
is a good enough solution in the current population. The last one was added
because I wasn't quite sure when the GA needed to stop in the beginning.

Some of these functions are abstracted away by traits, but still can be seen
as just fucntions.
\subparagraph{Step method}
The step method calls almost all given functions in the follwoing order:

\begin{enumerate}
	\item Select mates.
	\item Generate offspring.
	\item Select the fittest.
\end{enumerate}

\subparagraph{Genetic} hasGoodEnoughSolution is called in genetic,
which will continue calling steps until the population has a good enough
solution, or until parents and children are equal.
The population is a case class, so the generated comparison method is used
to check if the children and parents are equal.
The result of this function is a sequance of population. The first element
is te first generation and the last is the last generation.

\subparagraph{Start genetic} Start genetic makes a seq of the seed population and
calls genetic.


\paragraph{Evaluation} The evaluation object
contains the evaluation functions. The assignment asked to define four.
However the block evaluation (trap functions) can be implemented as the
same function with different values for the indices. This is done by
blockvaluation.

The create probe function is a utility function that allows the counting of
the fitness calls. The initial design didn't have a \emph{FitnessEvaluator},
but the assignment desired the counting of calls so I just wrapped the
function in a object that increases the count on every call. I had to
replace the function prototype in evolution with a trait because coutning
the calls of a function requires some form of state (an increasing variable)
I didn't know how to do that properly with just functions in scala yet,
so I just used a trait instead (trait = interface + state).

\paragraph{Offspring generator} The offspringGenerator implements the `how'
part of the pairedPopulations.  In there are three crossover functions
uniform, single point and two point crossover.

\subparagraph{Uniform crossover}
Uniform crossover checks character by character if they're equal
If not then a random one or zero is used. This is a strange
implementation, but its valid in this case because the only choice is
between 1 and 0, unequal means that there exist in both the father and
mother a one or zero. But not both. So we can just choose one.

\subparagraph{1 point crossover}
Generates a random number between zero and gene length. On the left side 
the father is used and on the right side the mother.

\subparagraph{2 point crossover}
Two point crossover was initially implemented
as first selecting a random value from 0 to gene length. The second would
be from first to gene length. However this is a biased method in that
on of the parents has more chance to be inherited. So
later I would just generate two random values. Between 0 and gene length
Then if the first that was generated is smaller then the second we use
the father as a base string and the mother as the other sequence. Otherwise
the other way around.

However the unbiased method performs structurally worse as can bee seen
in the experiment.

\subparagraph{Mutation}
Mutation isn't part of the offspringGenerator
because this is done by the member factory. It was easier to implement
in the member factory due to the architecture choices.

\label{paragraph:Fittest filter}
\paragraph{Fittest filter} The fittest filter decides how which of both
the parents and children continues.
The most simple implementation is the \emph{Killparents}, which
returns the children. \emph{Truncate elitism} selects the best of both parents
and children. \emph{Tournament elitism} matches child next to each parent
and decide which one is best. There is no kill children implementation
because nothing would ever change.

\paragraph{Mate selection} MateSelection implements tournament select.
This is the only mate selection algorithm implemented.

\paragraph{Member factories} The member factories are able to create
members from strings. These are either the tightly linked or randomly
linked.

\subparagraph{Mutation} Then the tightly or randomly linked
functions can also be wrapped in a mutation function,
which will mutate the string before passing it to the factory.
Its an decorator idea, only with functions.
\newpage

\subsubsection{Experiment}


\begin{figure}[ht!]
\centering
\begin{emp}[classdiag](20,20)
Class.ex("Experiment")(
	"+name:String",
	"+variation:String",
	"evolution: Evolution",
	"factory:String => IMember"
)(
	"+bisectionalSearch():StochasticRun",
	"verifyLowest(currentPop:Int,faults:Int,index:Int):Seq[RunResult]",
	"findOptimum(consideringSize:Int,difference:Int):Seq[RunResult]",
	"run(consideringSize:Int):RunResult"
);
Class.result("RunResult")(
	"+popSize:Int",
	"+success:Boolean",
	"+runtime:Long",
	"+generationCount:Int",
	"+fitnessCallCount:Int"
	)();
Class.stoch("StochasticRun")(
	"+runs:Seq[RunResult]",
	"+required:Int",
	)();
topToBottom.left(20)(ex,stoch);
leftToRight.top(20)(stoch,result);
drawObjects(ex,result,stoch);

link(association)(stoch.e -- result.w);
item(iAssoc)("*")(obj.ne = result.w);
\end{emp}
\caption{Experiment}
\end{figure}

The experiment is setup to do the bisectional search. The bisectional search
function will wrap the result of \emph{findOptimum} in a \emph{StochasticRun}.
\emph{findOptimum} is a recursive function that calls verify lowest to get a
collection of runResults. This is used to determine if the population size
should be increased or decreased like a bisection search.
\emph{verifyLowest} will try to run a population size 30 times, if the faults
limit is exceeded or the index exceeds the required run limit it will quit.
The run is considered to be successful if the count of successful run >= 
run limit - faults.

The run will execute the \emph{Evolution} that has been passed into this
experiment.  so this function will call startGenetic on the Evolution object, 
and put the results of the run into a \emph{RunResult}.

\subsection{Unit tests}
Because the assignment has very well defined behavior. I thought it was
a good Idea to use unit tests. To do this I use a library called
Scala test.

I found about 6 different bugs alone in the valuation functions by using
unit-testing, so I think this saved me considerable time. I also used this
to precisely define the behavior of 2 point cross.

By using unit tests I became very certain about the execution of my code.
It does precisely what its supposed to do even though I use functional
programming and it may look somewhat foreign.

\subsection{Build tool}
I use Gradle for building the project. Its the most advanced build platform
available and makes it easy to add dependencies to the project.
For example I use slf4j for logging.

Its also handy in that it comes with a self installing script that can be run
from any platform, Gradlew. So running the code I wrote on a different machine
becomes easy, Gradle makes sure the right dependencies are fetched.

\paragraph{Running it}

Open a shell and move to the project folder then type:

\lstset{language=Bash}
\begin{lstlisting}[frame=single]
	./gradlew run
\end{lstlisting}

\subsection{Libraries used}
In here I will describe why I chose to use these libraries
The details of the libraries can be found in the Gradle file (which
version and what maven group):

\begin{lstlisting}[frame=single]
	cat code/build.gradle
\end{lstlisting}


\begin{itemize}
	\item The Scala standard library, because I have to when using Scala.
	\item slf4j, a simple logging facade, this allows runtime switching of
		logging schemes, for example if you want to log into a database or
		a file instead of stdout you can do this relatively easy with slf4j
		This is not necessary right now but the API is simple enough to use
		everywhere.
	\item Scalaplot. A graph plotting library for Scala. I didn't want to
		do this myself, and this library supports SVG plotting and ASCII
		plotting.
	\item Scalatest. A testing library for Scala, it has some pretty advanced
		features but I use it mostly like JUnit.
\end{itemize}

\subsection{Speed up attempts}
I wanted my implementation to be fast, preferably faster than any other
implementation. I think I didn't succeed in this for the algorithm part
because I used a lot of slow mechanisms, such as functional combinators.
Although they don't increase the computational complexity,
they do certainly increase
the minimum required runtime. I'm quite sure that every combinator in sequence
is implemented as just another loop. However because I
run all experiments in parallel the total time may be faster. So in a way
I did succeed.

\subsubsection{Parallelization}
Since were doing string comparisons I had the idea of using parallelization to
speed up the algorithm by doing parallelization. This can be done very easily
in Scala with par collections (and the fact that immutability is default).

However once I implemented this, I came to the conclusion it was not faster.
The overhead for delegating and creating threads was more than the parallelization
gained.  After 10 minutes of waiting I decided to drop this idea
since without parallelization it took about 2 minutes.
The problem is that the evaluation function, even with gene length of 100, is
to cheap to make any reasonable parallelization effective.

So now all experiments are run at the same time. This probably increases the
average time of each `run' somewhat (not that much because the concurrent
collection does \href{http://docs.scala-lang.org/overviews/parallel-collections/configuration.html}{load balancing}).
It also drastically decreases the amount of time I have to wait. Which is quite
important, especially in the development phase.

The parallelization of experiments can be removed by deleting `par' on line 64
of Main.scala

\subsubsection{Evaluate each member on creation}
Initially my members just contained the string with the `genes'. This I passed
around and would evaluate when needed. However I realized that I could do this
evaluation eagerly on creation, to reduce the evaluation calls. A member has
to be evaluated anyways, so doing this on creation and just putting in the
evaluation saves some heavy evaluation operations.

\subsubsection{Only create children if the genes are new}
This optimization reduces the amount of evaluations required.  It becomes 
more effective at deeper generations, however it slows down the fastest 
GA's by a few milliseconds but the slower GA's  are quite a bit faster.

This can be done safely because strings are immutable and so are the members.
So I can reuse them whenever its needed.

I also had an idea of using hash-codes to reduce the amount of comparisons.
But it turns out that this increases the running time, so I quickly removed
this optimization again.

\section{Experiment}
In here I will discuss how the experiments ran and the thoughts I
above about the results.
First the setup will be discussed, since I added to the
original assignment out of curiosity. Then the hypothesis
will be discussed. After that the results will be shown
and finally I will discuss the conclusions I draw from the experiments.

\subsection{Experiment titles}
The experiment titles all follow the same structure:

\[F\mbox{ }L, E\]
Where F is the fitness function, L is the linkage and E is the elitism type.
The experiment results are grouped by this structure alphabetically.
The elitism types are discussed in the fittest fileter paragraph \ref{paragraph:Fittest filter}

Block is the trap function.

\subsection{Abbreviations}
\begin{tabular}{ll}
	Abrivation & Full name \\ \toprule
	2X & 2 Point cross \\
	2XB & 2 Point cross biased \\
	2XM & 2 Point cross mutated \\
	2XM & 2 Point cross mutated \\
	UX & uniform cross \\
	UXM & uniform cross mutated \\
	1X & 1 Point cross \\ \bottomrule
\end{tabular}

\subsection{Setup}
I added extra selection methods and crossing strategies to
the final experiment. I did this because I was curious how they would
compete with the original assignment.

After doing some sample experiments I found for example that 2x was going
up with number of successes in the biggest population size. So I wondered 
if doing my original interpretation of the select would work better.
I found that in almost every case this original interpretation is at least as
good or better. On both runtime and successes required population for success.

The biased two point cross will make 2 random values and use the lowest as
the base crossing for the same parent. The ` normal' 2 point cross will make
2 random values. If the first one generated is higher than the other one it
will make the father the mother.

It turns out however that the biased uniform cross is much more effective.
As it solves the deceptive trap function and leaves the `normal' one in
the dust. I actually expected an improvement when changing the biased
variant into the current variant. It was significantly worse however.
To show this I decided to include both.

1 point cross was added to see how that competes with 2 point cross.
I also used to have 2 point cross with
mutation but it wasn't significantly different then the normal 2X
and the graphs started to get cluttered so I decided to drop that in
favour of discussing the biased crossover.

Running the code now takes about 26 minutes in total with the
entire thing divided over 8 threads (quad core + hyper-threading). 

\subsection{Hypothesis}
The deceptive trap function will not be solved by any algorithm for
the population sizes. I think this because its to desirable to have
at least one 0000 block in the 100 string length.
Non-deceptive will be more successful. The others will definitively be solved.
I also think mutation will have an important effect and may even
allow deceptive to be solved.

The provided selection pressure would be about average.
It will probably always find a result.

\newpage
\input{result.tex}

\subsubsection{Graph observations}
The major drops are caused by an optimization I implemented. After 2 failed
run the stochastic insurance function will stop.

Often this happens when the bisectional search hones in on the optimum
This makes it appear as if there is a local gap, but its just
random chance.

\section{Conclusion}
My conclusion is that using both tournament selection and truncate elitism
is to harsh. It will slow down the GA and often not find a solution at all.
This happens for example with the deceptive trap function, randomly and
tightly linked.

The time vs population found ratio is also much better for using 
tournament select on both the elitism and selection.

Whats also interesting is that if you just throw away the parents and use
the children (kill parents), then you still will often perform better than
using tournament selection and truncate elitism (this is how the original
assignment was), although not as consistently as tournament elitism.

To my surprise deceptive trap was solved with decreased selection pressure
by both 2x and 1x, although 1x did worse. 

Mutation doesn't seem to have much effect UXM sometimes finds a better
solution but usually this is not the case. I initially thought mutation
would be very important but its just sort off random.  Mutation turned out
to have no significant effect other than just slowing things down.

Something to note is that my made up tournament elitism is actually in most
cases better, or at least equal to the truncate elitism. So when I saw
this I tried using my other initially used algorithm where I would completely
ignore the parents (kill the parents). However it appears this does not
perform better.

Two point cross and single cross both seem to find results in deceptive
random tournament elitism.

Linearly scaled and uniformly scaled are very easy to solve for almost all
algorithms. Although the 2x and 1x are at a big disadvantage compared to UX.

\end{empfile}
\end{document}


