\documentclass{article}
\usepackage{txfonts}
\usepackage{booktabs}
\usepackage{color}
\usepackage{bussproofs}
\usepackage{graphicx}
\usepackage{pifont}
\usepackage{qtree}
\usepackage{tikz}
\usepackage{listings}
\usepackage{hyperref}
\newenvironment{scprooftree}[1]%
{\gdef\scalefactor{#1}\begin{center}\proofSkipAmount \leavevmode}%
{\scalebox{\scalefactor}{\DisplayProof}\proofSkipAmount \end{center} }


\newcommand{\brcell}[2][l]{%
	\begin{tabular}[#1]{@{}l@{}}#2\end{tabular}}
\begin{document}
\author{Jappie Klooster}
\title{Mutable gene possibilities}
\maketitle
#Permutation problems

Clasical example is traveling salesman.
non-banary strings, will generate many illegal soltions.
possible solutions:
	Alternative search space representation.

# Mutation operators
best one depends on the problem.

\begin{itemize}
	\item absolute position
	\item adjencency
	\item relative ordering
\end{itemize}

## Swap and insert
ABCDEFGH
	insert: ABDEFCGH
	swap:	ABGDEFCH

## Two opt
Randomly select two points along the sequance and invort one of the
subsequences.

\begin{prooftree}
	\AxiomC{AB|CDE|FG}
	\UnariInfC{AB|EDC|FG}
\end{prooftree}

Adjencency is importent

## Recombination
Capital and lower laters are the same elements.

### Order crossover
Select 2 random crosspoints
Copy the subseqent croproints from p1
after the second crosspoints try to fill in the missing elements
form p2 in relative order

### Partially mapped crossofer
Select 2 random crosspoints
Copy p2 to child
Copy elements between random points to p2, for each element
copied between random points replace the crossed out point
with the newly coppied element

### Position crossover

### Maximal preservative crossover
Use a lot of adjencency information, fallback to mapped crosver if it doesn't
work??

### Cycle crossover
Check which elements are connected to each other like a binary tree in a
int array

ABCDEFGHI
fcdaebhig
111121333

1: A->f->F->b->B->c->C->d->A->a->A
2: E->e
3: G->h->H->i->I->g

cross the full cycles, so choose between any of the 1,2 or 3

## Edge recombination
Parent tours...

City	Edges
A		B F C E
B		A C D F
C 	 	B D A
D 	 	C E B
E 	 	D F A
F 	 	A E B

Random choice => B
next candiadates: ACDF

count remaining options, pick the lowest C
Next choose D because only 1
E -> A -> F

child: B -> C -> D -> E -> A -> F

But the child has the problem that it didn't inherit D->F

# Fitness correlation coefficients
(S_1, F_1) \to (S_1, F_1)
(S_2, F_2) \to (S_2, F_2)
\vdots
(S_{n-2}, F_{n-2}) \to (S_{n-2}, F_{n-2})
(S_{n-1}, F_{n-1}) \to (S_{n-1}, F_{n-1})

##crossover
(S_1, F_1) \cross (S_1, F_1)
(S_2, F_2) \cross (S_2, F_2)
\vdots
(S_{n-2}, F_{n-2}) \cross (S_{n-2}, F_{n-2})
(S_{n-1}, F_{n-1}) \cross (S_{n-1}, F_{n-1})

##Corolotion coefition $\rho$
0 = a cloud
-1 = anti evolution, natural deselection
1 = evolution, natural selection

$-1 \steq \rho \bteq 1$

### Traveling salesman TSP
Edge crossover was made for TSP
it has .9, which is very high.

### Alternative
Just run the thing and see which one is faster.

# Evolutionary strategies (ES)
ES use an advanced mutation operator which controls its own mutability
self adaptation.

This works by using gausion distrobutions, it should adapt to the topology
of the landscape. If its to big then its like random search. If its to
small it can get stuck at a local optimum.

This gausian uses more information about the problem, you know information
so you can reduce the search space.

## ES representation

\end{document}
