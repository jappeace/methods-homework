\documentclass{article}
\usepackage{amsmath}
\usepackage{txfonts}
\usepackage{booktabs}
\usepackage{color}
\usepackage{bussproofs}
\usepackage[pdftex]{graphicx}
\DeclareGraphicsRule{*}{mps}{*}{}
\usepackage{pifont}
\usepackage{qtree}
\usepackage{emp}
\usepackage{listings}
\usepackage{hyperref}
\usepackage{svg}
\newenvironment{scprooftree}[1]%
{\gdef\scalefactor{#1}\begin{center}\proofSkipAmount \leavevmode}%
{\scalebox{\scalefactor}{\DisplayProof}\proofSkipAmount \end{center} }


\newcommand{\brcell}[2][l]{%
	\begin{tabular}[#1]{@{}l@{}}#2\end{tabular}}
\begin{document}
\begin{empfile}
\begin{empcmds}
input metauml;
\end{empcmds}
\author{Jappie Klooster}
\title{A practical look at genetic algorithms}
\maketitle

\section{Introduction}
This is a report of the evolutionary practical assignment 1.
In here I will discuss the execution of the assignment, what it was, which
methodology was used to execute the assignment.

The intended audience is people familiar with programming.

\section{Assignment}
The assignment was to implement the following functions and algorithms and
then do experimentation with help of these.

\subsection{Functions}
\subsubsection{Counting functions}
\[x_i \in \{0,1\}:CO(x_1\dots x_l)=\sum^l_{i=1}x_i\]
\[x_i \in \{0,1\}:CO(x_1\dots x_l)=\sum^l_{i=1}i x_i\]
\subsubsection{Trap function}
\[TF(x_1\dots x_l) = \sum^{\frac{l}{k}-1}_{j=0}B(x_{jk_1}\dots x_{jk+k})\]

with:

\[B(x_1\dots x_k)=\left\{
	\begin{matrix}
		k & \mbox{ if } &   CO(x_1\dots x_k) = k \\
		k - d - \frac{k-d}{k-1} CO(x) & \mbox{ if } & CO(x_1\dots x_k) < k \\
	\end{matrix}
\right. \]

\subsection{Experiment}
After implementing these functions we were tasked to make an experiment of it
and make some graphs of the results and draw your conclusions.

\section{Methodology}
The developing method I chose to use is what I like to call the headless
chicken approach. In here you just start implementing functionalities in
hope the result will be good enough. I think this is preferable to other
methodologies such as PRINCE2 or SCRUM because it has very little
methodology overhead.  With a precise defined assignment like this one I
think reducing overhead is one of the main concerns.
Besides most professional methodologies are more
aimed at making teams works rather than individuals.

\subsection{Language choice}
I chose to use Scala to implement this project. The main reason being that I've
played with Scala before but never have done anything constructive with it.

Scala is both a functional and an object orientated program that runs on the JVM
It compiles to byte-code and is statically typed.

It also can use any Java library and integrates seamlessly with Java, (although
calling methods the methods of a Scala project from a Java project can be
a little clumsy sometimes, because Scala supports operator overloading in
a strange way).

Other options I considered were:
\begin{itemize}
	\item Java, which I'm very familiar with but I find quite boring to use.
		Its very verbose.
	\item C\#, the language is good but setting it up is difficult since I
		use Linux, it also has no Gradle support which makes difficult to
		use separate libraries or you have to rely on {.}net libraries only
		which is limited. It also prefers object orientated rather than
		functional programming, Scala prefers functional over object
		orientated. I'm off the opinion that this assignment is quite good
		to try functional programming on. Scala's syntax is also more concise.
		In fact C\# feels pretty dated which it flamboyant use of `;' and the
		strange semantic difference that exists between class brackets and
		method brackets. Scala's type system is also more advanced than
		C#'s, so its in a way more object orientated.
	\item Rust/Go, I'm curious about both these languages, but I'm not at
		all familiar with them, so I probably should do first a little
		hobby project with them before starting doing something serious like
		this, I've already done this with Scala before.
	\item Python, I think the project is in that grey area where weakly
		typed langauges become less powerful in use. Especially considering
		the fact how quickly these algoritms blow up in runtime. Static type
		checking then becomes a gift form heaven.
	\item C++, I don't want to deal with C++.
\end{itemize}

\subsection{Architecture}
%TODO: check spell from here
The code is split up into 4 parts, first there are the case classes. These are
immutable classes that hold the domain information (Java beans).
Then there are the algorithm classes and objects. These are pretty straight
forward logical structures. They implement the atoms of possible GA's, such
as for example uniform crossover. After that comes the experiment and the result
classes, the experiment just uses the GA in a specific way.
Finally there is the presentation object \emph{Main}. the only thing it does
is interpret the experiment into this case into a latex structure. There is no
diagram discussing Main, because there is little architecture involved.

\begin{figure}[ht!]
\centering
\begin{emp}[classdiag](20,20)
Class.mem("Member")("+Fitness:Int", "+genes:String")();
Class.pop("Population")("+members:Seq[Member]", "+factory:String=>Member")();
Class.pairedPop("PairedPopulation")("+members:Seq[Pair]","+factory:String=>Member")(
"+pairedToPop(how):Population");
Class.pai("Pair")("+father:Member","+mother:Member")();

Instance.population("Population")(
"createOnesAndZeros(factory:String=>Member,popCount:Int,geneLength:Int):Population"
);

leftToRight.top(20)(pop,pairedPop);
topToBottom.left(20)(population, pop);
topToBottom.left(60)(pop,mem);
leftToRight.top(80)(mem, pai);

drawObjects(mem,pop,pairedPop,pai,population);
link(association)(pai.n -- pairedPop.s);
item(iAssoc)("*")(obj.se = pai.n);
link(association)(mem.e -- pai.w);
item(iAssoc)("2")(obj.sw = mem.e);

link(association)(mem.n -- pop.s);
item(iAssoc)("*")(obj.se = mem.n);
\end{emp}
\caption{Case classes}
\end{figure}

The case classes add meaning to the structures.
A case class is a Scala concept were you just define the attributes. Then
Scala generates the `equals', `hashCode' and `toString' methods for you.
They're also by default immutable, which is something I kept intact. You
can make mutations by using the generated copy function, in which you can
specify what is changed.

A \emph{Member} is something with a gene. The fitness value gets pre calculated by
the factory, so that every members' fitness is only calculated once. This can
be done reliably because the member is immutable.

The members are generated in batches into a population, either by the paired
population or by the Population companion object which provides a
`createOnesAndZeros' function.
A population will also have a function with it to generate new members.
The population class doesn't use it itself, but when its members are paired
up into \emph{Pair}'s. A \emph{PairedPopulation} can use the creation function
of the original population to create new members.

The \emph{PairedPopulation} has also a utility function pairedToPop,
this receives a how, in which it can
insert zipped genes (char by char paired up strings) and from which comes
a char sequence. This how is implemented by one of the crossing methods.
This abstraction was added because the code around the cross function was
always the same.
The resulting string from `how' is then inserted into a member
factory to create a new member (the member factory then allows mutation
or obfuscation of the new genes).

\begin{figure}[ht!]
\centering
\begin{emp}[classdiag](20,20)
Class.evo("Evolution")(
"+valuation: FitnessEvaluator",
"mateSelector:IMateSelector",
"offSpringGenerator: PairedPopulation => Population",
"fittestFilter: (Population, Population) => Population",
"hasGoodEnoughSolution: Population => Boolean")(
"startGenetic(seed:Population): Seq[Population]",
"genetic(prev:Seq[Population]): Seq[Population]",
"step(parents:Population): Population",
);
Instance.fit("Fitness")(
	"uniformlyScaled()",
	"linearlyScaled()",
	"blockValuation(block:Seq[Float])",
	"createProbe(function):FitnessEvaluator"
);
Instance.filt("FittestFilter")(
	"killParents()",
	"truncateElitism()",
	"tournementElitism()"
	);
Instance.select("MateSelection")(
	"tournementSelect()"
);
Instance.factor("MemberFactories")(
	"tightlyLinkded()",
	"randomlyLinked()",
	"withCoinflipMutation(function):function",
);
Instance.off("OffspringGenerator")(
	"uniformCross()",
	"singlePointCross()",
	"twoPointCross()"
);

topToBottom.left(20)(evo,fit,filt,factor);
leftToRight.top(20)(fit,off);
leftToRight.top(20)(filt,select);

drawObjects(evo,fit,factor,filt,select,off);
\end{emp}
\caption{Algorithm}
\end{figure}

The evolution class is the common structure of a GA{.} It's very simple and
basically follows these steps:
\begin{enumerate}
	\item Select mates.
	\item Generate offspring.
	\item Select the fittest.
\end{enumerate}

To construct an Evolution you need to pass a \emph{valuation function},
a \emph{mateselector}, An \emph{offspring generator}, 
a \emph{fittestFilter}. And a function that determines if there
is a good enough solution in the current population. The last one was added
because I wasn't quite sure when the GA needed to stop in the beginning.

The step method calls almost all given functions in above order except for has good
enough solution. This function is called in genetic, which will continue
calling steps until the population is good enough, or until parents
and children are equal. The population is a case class, so
the generated comparison method is used to check if the children and parents
are equal. \emph{Start genetic} makes a seq of population and calls genetic.

The objects defined above all have functions with the types same named evolution
attributes. So the functions are just basically options you can use to create an
evolution from. There are some exceptions which allow wrapping is the
withCoinFlipMutation function in
memberFactories. This function acts as a decorating function. So you can wrap
it around another function to create a new function.

Objects are just singletons, or Java static space. I use the object keyword in
Scala instead of class So I chose also to use an object in the UML
representation. Scala seems to stick pretty close to UML notation anyways.
The objects don't have types however, because they are the types. They're
language enforced singletons.

Fitness contains the evaluation functions. The assignment defined four. However
the block evaluation (deceptive/nondeceptive) can be implemented as the
same function with different values for the indices (which blockvaluation 
does). Interesting is that the blockvaluation in code is shorter than the
mathematical definition.
The create probe function is a utility function that allows the counting of
the fitness calls. The initial design didn't have a \emph{FitnessEvaluator},
but the assignment desired the counting of calls so I just wrapped the
function in a object that increases the count on every call.

The offspringGenerator implements the 'how' part of the pairedPopulations.
The mutation isn't part of the offspringGenerator because this is done by
the member factory.

The fittest filter decides how which of both the parents and children 
continues. The most simple implementation is the Killparents, which
returns the children. Truncate elitism selects the best of both parents
and children. Tournament matches child next to each parent
and decide which one is best. There is no kill children implementation
because nothing would ever change.

MateSelection implements tournament select. This is the only mate selection
algorithm implemented currently.

The member factories are able to create members from strings. These are either
the tightly linked or randomly linked. Then the tightly or randomly linked
functions can also be wrapped in a mutation function,
which will mutate the string before passing it to the factory.
Its an decorator idea, only with functions.

\begin{figure}[ht!]
\centering
\begin{emp}[classdiag](20,20)
Class.ex("Experiment")(
	"+name:String",
	"+variation:String",
	"evolution: Evolution",
	"factory:String => IMember"
)(
	"+bisectionalSearch():StochasticRun",
	"verifyLowest(currentPop:Int,faults:Int,index:Int):Seq[RunResult]",
	"findOptimum(consideringSize:Int,difference:Int):Seq[RunResult]",
	"run(consideringSize:Int):RunResult"
);
Class.result("RunResult")(
	"+popSize:Int",
	"+success:Boolean",
	"+runtime:Long",
	"+generationCount:Int",
	"+fitnessCallCount:Int"
	)();
Class.stoch("StochasticRun")(
	"+runs:Seq[RunResult]",
	"+required:Int",
	)();
topToBottom.left(20)(ex,stoch);
leftToRight.top(20)(stoch,result);
drawObjects(ex,result,stoch);

link(association)(stoch.e -- result.w);
item(iAssoc)("*")(obj.ne = result.w);
\end{emp}
\caption{Experiment}
\end{figure}

The experiment is setup to do the bisectional search. The bisectional search
function will wrap the result of \emph{findOptimum} in a \emph{StochasticRun}.
\emph{findOptimum} is a recursive function that calls verify lowest to get a
collection of runResults. This is used to determine if the population size
should be increased or decreased like a bisection search.
\emph{verifyLowest} will try to run a population size 30 times, if the faults
limit is exceeded or the index exceeds the required runs limit it will quit.
In the latter case the run is considered to be successful.
The run will execute the GA search according to the configuration of the 
experiment, so this function will call startGenetic, and put the results
of the run into a RunResult. The RunResult will actually truncate a lot of
information. Not the entire resulting population is stored. It wasn't
necessary for the experiment and I was concerned about memory. Or rather
I didn't want to be too wasteful, especially considering that the JVM
reserves big blocks of memory.

\clearpage


Most of the GA architecture was already in place before the assignment became
available. I did this based on the notes I took in class. When the assignment
became available I was almost completly done with the GA{.} There were some bugs
in my code however and I interperted some things wrong. I also didn't write
anything for an experiment. So I still needed to do that.

\subsubsection{Unit tests}
Because the assignment has very well defined behavior. I thought it was
a very wise Idea to use unit tests. To do this I use a library called
Scala test.

I found about 6 different bugs alone in the valuation functions by using 
unit-testing, so I think this saved me considerable time.
\subsubsection{Build tool}
I use Gradle for building the project. Its the most advanced build platform
available and makes it easy to add dependencies to the project.
For example I use slf4j for logging.

Its also handy in that it comes with a self installing script that can be run
from any platform, Gradlew. So running the code I wrote on a different machine
becomes easy, Gradle makes sure the right dependencies are fetched.

\paragraph{Running it}

Open a shell and move to the project folder then type:

\lstset{language=Bash}
\begin{lstlisting}[frame=single]
	./gradlew run
\end{lstlisting}

\subsubsection{Libraries used}
In here I will describe why I chose to use these libraries
The details of the libraries can be found in the gradle file (which
version and what maven group):

\begin{lstlisting}[frame=single]
	cat code/build.gradle
\end{lstlisting}


\begin{itemize}
	\item The Scala standard library, because I have to when using Scala.
	\item slf4j, a simple logging facade, this allows runtime switching of
		logging schemes, for example if you want to log into a database or
		a file instead of stdout you can do this relatively easy with slf4j
		This is not necessary right now but the API is simple enough to use
		everywhere.
	\item Scalaplot. A graph plotting library for Scala. I didn't want to
		do this myself, and this library supports SVG plotting and ASCII
		plotting.
	\item Scalatest. A testing library for Scala, it has some pretty advanced
		features but I use it mostly like JUnit.
\end{itemize}

\subsection{Speed up attempts}
I wanted my implementation to be fast, preferably faster than any other
implementation, now I think I didn't succeed in this because I used a
lot of slow mechanisms, such as functional combinators, and although
they don't increase the computational complexity, they do certainly increase
the minimum required runtime, I'm quite sure that every combinator in sequence
is implemented as just another loop. However because I
run all experiments in parallel the total time may be faster.
\subsubsection{Parallelization}
Since were doing string comparisons I had the idea of using parallelization to
speed up the algorithm by doing parallelization. This can be done very easily
in Scala with par collections (and the fact that immutability is default).

However once I implemented this, I came to the conclusion it was not faster.
The overhead for delegating and creating threads was more than the parallelization
gained.  After 10 minutes of waiting I decided to drop this idea
since without parallelization it took about 2 minutes (I tried this
on just one run, so it shouldn't take that long).

The problem is that the evaluation function, even with gene length of 100, is
to cheap to make any reasonable parallelization effective. However what was effictive
is the parallelization of the experiments. This reduced total runtime
significantly. However this is not seen in the statistics.

So now all experiments are run at the same time. This probably increases the
average time of each `run' somewhat (not that much because the concurrent
collection does \href{http://docs.scala-lang.org/overviews/parallel-collections/configuration.html}{load balancing}).
It also drastically decreases the amount of time I have to wait. Which is quite
important, especially in the development phase. (although I also used
other tricks, such as instead of 30, just doing 3 runs)

The parallelization of experiments can be removed by deleting `par' on line 64
of Main.scala

\subsubsection{Evaluate each member on creation}
Initially my members just contained the string with the `genes'. This I passed
around and would evaluate when needed. However I realized that I could do this
evaluation eagerly on creation, to reduce the evaluation calls. A member has
to be evaluated anyways, so doing this on creation and just putting in the
evaluation saves some heavy evaluation operations.

\subsubsection{Only create children if the genes are new}
If the genes are new then create the children otherwise return the parent.
This reduces evaluations and object creations. Its becomes more effective
at deeper generation, it even slows down the fastest GA's by a few milliseconds
but the deeper GA are lots faster.

This can be done safely because strings are immutable and so are the members.
So I can reuse them whenever its needed.

To reduce the amount of string comparisons, first the hash code function is called
This will cache itself, I know this because I inspected the source code of the
String class in Java.

I was very proud for coming up with  this idea, but when I ran the experiment with
and without hash code I came to the conclusion that first doing a hash code operation
significantly *slows* execution. This becomes worse as population size increases.

\section{Experiment}
Something to note is that my made up tournament elitism is actually in most
cases better, or at least equal to the truncate elitism. So when I saw
this I tried using my other initially used algorithm where I would completely
ignore the parents (kill the parents). However it appears this does not
perform better.

I also added two point cross with mutation and 1 point cross to see how
those fair, running the code now takes about 40 minutes in total with the
entire thing divided over 8 threads (quad core + hyper-threading). The only
other improvement I could think of is running it on the GPU, but I think
the algorithm is to complex for that and I don't know how to do that,
especially considering I'm using the JVM.

Two point cross and single cross both seem to find results in deceptive
random tournament elitism.

\subsection{Hypothesis}
The deceptive trap function will not be solved by any algorithm.
non-deceptive may be more successful. The others will be solved.
I also think mutation will have an important effect and may even
allow deceptive to be solved.

I also think that the provided selection pressure would be about
average. It will probably always find a result.

\subsection{Execution}
Note that I added some extra selection methods and crossing strategies to
the final experiment. I did this because I was curious how they would
compete with the original assignment.

After doing some sample experiments I found for example that 2x was going
up with number of successes in the biggest population size. So I wondered 
if doing my original interpretation of the select would work better.
I found that in almost every case this original interpretation is at least as
good or better. On both runtime and successes required population for success.

\newpage
\input{result.tex}

\section{Conclusion}
%What did I learn from my observation?
My conclusion is that using both tournament selection and truncate elitism
is to harsh. It will slow down the GA and often not find a solution at all
(in case of deceptive trap, 2x should've come up with something). the 
time vs population found ratio is also much better for just using tournament
select.

Whats also interesting is that if you just throw away the parents and use 
the children (kill parents), then you still will often perform better than
using tournament selection and truncate elitism, although not as 
consistently as tournament elitism.

To my surprise deceptive trap was solved with decreased selection pressure
by both 2x and 1x, although 1x did worse. Also adding mutation doesn't do
the algorithm much good. 2XM and UXM consistently find worse minimum populations
than respectively 2X and UX{.}
This probably means the mutation is too disruptive.

Mutation turned out to have no significant effect other than just slowing
things down.

\end{empfile}
\end{document}

