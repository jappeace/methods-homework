\documentclass[10pt,landscape]{article}
\usepackage{multicol}
\usepackage{calc}
\usepackage{ifthen}
\usepackage[landscape]{geometry}
\usepackage{amsmath,amsthm,amsfonts,amssymb}
\usepackage{color,graphicx,overpic}
\usepackage{hyperref}


\pdfinfo{
  /Title (example.pdf)
  /Creator (TeX)
  /Producer (pdfTeX 1.40.0)
  /Author (Seamus)
  /Subject (Example)
  /Keywords (pdflatex, latex,pdftex,tex)}

% This sets page margins to .5 inch if using letter paper, and to 1cm
% if using A4 paper. (This probably isn't strictly necessary.)
% If using another size paper, use default 1cm margins.
\ifthenelse{\lengthtest { \paperwidth = 11in}}
    { \geometry{top=.5in,left=.5in,right=.5in,bottom=.5in} }
    {\ifthenelse{ \lengthtest{ \paperwidth = 297mm}}
        {\geometry{top=1cm,left=1cm,right=1cm,bottom=1cm} }
        {\geometry{top=1cm,left=1cm,right=1cm,bottom=1cm} }
    }

% Turn off header and footer
\pagestyle{empty}

% Redefine section commands to use less space
\makeatletter
\renewcommand{\section}{\@startsection{section}{1}{0mm}%
                                {-1ex plus -.5ex minus -.2ex}%
                                {0.5ex plus .2ex}%x
                                {\normalfont\large\bfseries}}
\renewcommand{\subsection}{\@startsection{subsection}{2}{0mm}%
                                {-1explus -.5ex minus -.2ex}%
                                {0.5ex plus .2ex}%
                                {\normalfont\normalsize\bfseries}}
\renewcommand{\subsubsection}{\@startsection{subsubsection}{3}{0mm}%
                                {-1ex plus -.5ex minus -.2ex}%
                                {1ex plus .2ex}%
                                {\normalfont\small\bfseries}}
\makeatother

% Define BibTeX command
\def\BibTeX{{\rm B\kern-.05em{\sc i\kern-.025em b}\kern-.08em
    T\kern-.1667em\lower.7ex\hbox{E}\kern-.125emX}}

% Don't print section numbers
\setcounter{secnumdepth}{0}


\setlength{\parindent}{0pt}
\setlength{\parskip}{0pt plus 0.5ex}

%My Environments
\newtheorem{example}[section]{Example}
% -----------------------------------------------------------------------

\begin{document}
\raggedright
\footnotesize
\begin{multicols}{3}


% multicol parameters
% These lengths are set only within the two main columns
%\setlength{\columnseprule}{0.25pt}
\setlength{\premulticols}{1pt}
\setlength{\postmulticols}{1pt}
\setlength{\multicolsep}{1pt}
\setlength{\columnsep}{2pt}

\begin{center}
     \Large{\underline{Evolutionary Computing}} \\
\end{center}

\section{Genetic algorithms}
\subsection{Schema}
$o(h)$: schema order $o(11\#\#0)$ ie nonwildcard \# elements.
$\delta(h)$: schema defining length, maximum distance between two defining
symbols. $m(h,t)$: number of schema h instances at generation t.
$f(h,t)=\sum_{i\in P}f_i$ schema fitness is average fitness of members.
\subsubsection{Growth by selection}
reproduction ratio $\phi(h,t)=\frac{m(h,t^s)}{m(h,t)}$.
\paragraph{Proportionate selection}
$f_i$ fitness ind i. Probability individiual i selected: $\frac{f_i}{\sum f_i}$.
$N$ pop size. Expected number of copies
$\frac{f_i}{\sum f_i} \cdot N = \frac{f_i}{f(t)}$.
Expected number of copies of schema $h$ members:
$m(h,t^s) = m(h,t)\phi(h,t)=m(h,t)\frac{f(h,t)}{f(t)}$.
\paragraph{Tournament selction}
tournament size $s: 0 \leq \phi(h,t) \leq s$.

\subsubsection{Distruption by mutation}
probability bit flipped: $p_m$.
schema $h$ survives iff all the bit values are \emph{not} mutated
$p_{survival}=(1-P_m)^{o(h)}$. for small values $p_m << 1$,
$(1-p_m)^{o(h)}\approx 1 - o(h)\cdot p_m$.
disruption factor $\epsilon(h,t)$ by mutation: $\epsilon(h,t)=o(h)\cdot P_m$.

\subsubsection{Distruption by mutation}
Probablity crossover applied $P_c$. Bit swap probability: $p_x$.
\paragraph{1-point crossover} schema $h$ survives iff cutpoint \emph{not}
withing defining length $\delta$: $p_{survival} = 1-\frac{\delta(h,t)}{l-1}$
\paragraph{Uniform crossover} Schema $h$ survives iff none or all bits
swapped together. $p_{survival}=p^{o(h)}_x + (1-p_x)^{o(h)}$.

\paragraph{Distruption factor} $\epsilon(h,t)$ by recombination:
$\epsilon(h,t)=P_c\cdot(1-p_{survival})$

\subsubsection{Theorom}
Selection mutation and recombination combined:
$m(h,t+1)\leq m(h,t)\phi(h,t)[1-\epsilon(h,t)]$.
Net growth factor: $\gamma(h,t)=\frac{m(h,t+1)}{m(h,t)}$.

\subsection{Fitness Correlation coefficents}
$F_p$: mean fitness of the parents. $F_c$: mean fitness of the children.
$\sigma(F_p) =$ standard deviation of fitness parents
$\sigma(F_c) =$ std. dev. of children. coveriance fitness between parents and children:
$cov(F_p,F_c)=\sum^n_{i=1}\frac{(f(p_{gi})-F_p)(f(c_{gi})-f_c)}{n}$.
Operator fitness correlation coeficient
$\rho_{op}=\frac{cov(F_p,F_c)}{sigma(F_p)\sigma(F_c)}$.

\section{ES representation}
Fitness function: $f(x_1,\dots,x_n):\mathbb{R}^n\to \mathbb{R}$.
Genotype representation of an individual solution:
$(x_1,\dots,x_n,\sigma^2_1,\dots,\sigma^2_n, c_{12},\dots,c_{n-1,n})$,
paremeters $(x_1, \dots,x_n)$ need to be optimized.
Individiual solution consists of 3 parts: $\vec{x}$: problem vairables
$\Rightarrow$ Fitness $f(\vec{x})$. $\vec{sigma}$: standard deviations
$\Rightarrow$ variances. $\vec{\alpha}$: rotation angles $\Rightarrow$
covariances.

\end{multicols}
\end{document}
