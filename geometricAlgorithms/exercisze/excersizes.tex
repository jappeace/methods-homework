\documentclass{article}
\usepackage{amsmath}
\usepackage{txfonts}
\usepackage{booktabs}
\usepackage{color}
\usepackage{bussproofs}
\usepackage[pdftex]{graphicx}
\DeclareGraphicsRule{*}{mps}{*}{}
\usepackage{pifont}
\usepackage{qtree}
\usepackage{emp}
\usepackage{listings}
\usepackage{hyperref}
\usepackage{svg}
\usepackage{pgfplots}
\usepackage{pgfplotstable}
\usepackage{seqsplit}
\usepackage{algorithmicx}
\usepackage{algpseudocode}
\pgfplotsset{
    compat=1.9,
    compat/bar nodes=1.8
}

\newenvironment{scprooftree}[1]%
{\gdef\scalefactor{#1}\begin{center}\proofSkipAmount \leavevmode}%
{\scalebox{\scalefactor}{\DisplayProof}\proofSkipAmount \end{center} }


\newcommand{\brcell}[2][l]{%
	\begin{tabular}[#1]{@{}l@{}}#2\end{tabular}}

% "define" Scala
\lstdefinelanguage{scala}{
  morekeywords={abstract,case,catch,class,def,%
    do,else,extends,false,final,finally,%
    for,if,implicit,import,match,mixin,%
    new,null,object,override,package,%
    private,protected,requires,return,sealed,%
    super,this,throw,trait,true,try,%
    type,val,var,while,with,yield},
  otherkeywords={=>,<-,<\%,<:,>:,\#,@},
  sensitive=true,
  morecomment=[l]{//},
  morecomment=[n]{/*}{*/},
  morestring=[b]",
  morestring=[b]',
  morestring=[b]"""
}
\begin{document}
\begin{empfile}
\begin{empcmds}
input metauml;
\end{empcmds}
\author{Jappie Klooster}
\title{Stay in Shape! A geometric algorithmic excersize collection}
\maketitle

\section{Introduction}
Referencing he book Computational geometry algorithms and application.
This collection  tries to solve a selection of the excersizes in that book.

Referer to this work with care as its meant to be homework (although
I try to make it correct).

\section{The intersection detection problem}
The intersection detection problem for a set S of n line segments is to
determine whether there exists a pair of segments in S that intersect. Give
a plane sweep algorithm that solves the intersection detection problem in
$O(n \cdot  log_2(n))$ time.

\begin{algorithmic}
	\State Sort all line segment $s \in S$ end points $p$ based on
		$y$ coordinate.
		$p$ should have a reference to its coorisponding point (ie the start
		point has a reference to the end point and vice versa). The resulting
		collection is called $E_{sorted}$
	\For{point$\in E_{sorted}$}
		\State
		\If{begin point} insert in State. On inserted point calculate
		if there is a crossing if so insert in $E_{sorted}$.
		\EndIf
		\If{end point} remove begin point from state.
		\EndIf
		\If{Crossing point} add to result.
		\EndIf
	\EndFor
\end{algorithmic}

\end{empfile}
\end{document}
