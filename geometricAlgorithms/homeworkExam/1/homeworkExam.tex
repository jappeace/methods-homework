\documentclass{article}
\usepackage{amsmath}
\usepackage{algorithmicx}
\usepackage{algpseudocode}
\usepackage{pst-solides3d}
\usepackage{auto-pst-pdf}
\usepackage{tikz}

\begin{document}
\author{Jappie Klooster}
\title{Examination of shapes! (homework exam)}
\maketitle

\section{Polyhedron}

\subsection{When is the $xy$-silhouette of a convex polyhedron $P$ is unique}

Assuming uniqueness in this case means an $xy$-silhouette that can only be
generated from a polyhedron P when looking at it from a specific angle.

So when projecting a polyhedron P on an $xy$ plane (ie removing the z values).
You need to have 3 angles. $x$ rotation, $y$ rotation and $z$ rotation.

An $xy$-silhouette is said to be unique if there is only one set of rotations
that can generate that silhouette.

\subsubsection{Not unique}
A good and simple example of not unique can be seen in figure
\ref{fig:Not unique} a tetrahedron.

\begin{figure}[h]
\begin{pspicture}(-2,-2)(3,3)
\psset{viewpoint=100 0 30,Decran=100}
\psSolid[object=tetrahedron,
		r=2,
		RotZ=65,
		fillcolor=magenta!20,
        action=draw**,
		numfaces=all]%
\psSolid[object=tetrahedron,
		r=2,
		RotZ=125,
		fillcolor=magenta!20,
        action=draw**,
		numfaces=all](0, 4, 0)%
\psSolid[object=tetrahedron,
		r=2,
		RotZ=185,
		fillcolor=magenta!20,
        action=draw**,
		numfaces=all](0, 8, 0)%
\end{pspicture}


\label{fig:Not unique}
\caption{Not unique}
\end{figure}
\newpage

\subsubsection{Unique}
From the unique tetrahedron I tried making a not unique one, I did
this initially with blender but later recreated it with the pst-solides3d
latex package as can be seen in figure \ref{fig:Unique} a malformed
tetrahedron. The figure will always start from the same side and apply
rotation on different axis.
No matter how you rotate the tetrahedron the xy-silhoutte will look
different until you finish the full rotation.


\begin{figure}[h]
\begin{pspicture}(-2,-2)(3,3)
\psset{viewpoint=100 0 0,Decran=100}
\psSolid[object=new,
        action=draw**,
		fillcolor=cyan,
		numfaces=all,
        sommets=
		0 0 2
		-1.5 -0.1 0
		1 -1 0
		1.5 2.5 0,
   faces={
       [0 1 2]
       [0 2 3]
       [0 3 1]
       [1 3 2]}](0,0,0)%
\psSolid[object=new,
        action=draw**,
		fillcolor=cyan,
		RotZ=65,
		numfaces=all,
        sommets=
		0 0 2
		-1.5 -0.1 0
		1 -1 0
		1.5 2.5 0,
   faces={
       [0 1 2]
       [0 2 3]
       [0 3 1]
       [1 3 2]}](0,4.5,0)%
\psSolid[object=new,
        action=draw**,
		fillcolor=cyan,
		RotZ=125,
		numfaces=all,
        sommets=
		0 0 2
		-1.5 -0.1 0
		1 -1 0
		1.5 2.5 0,
   faces={
       [0 1 2]
       [0 2 3]
       [0 3 1]
       [1 3 2]}](0,9,0)%
	   %%%%%%%%%%%%%%%% X
\psSolid[object=new,
        action=draw**,
		fillcolor=cyan,
		numfaces=all,
        sommets=
		0 0 2
		-1.5 -0.1 0
		1 -1 0
		1.5 2.5 0,
   faces={
       [0 1 2]
       [0 2 3]
       [0 3 1]
       [1 3 2]}](0,0,-4)%
\psSolid[object=new,
        action=draw**,
		fillcolor=cyan,
		RotX=65,
		numfaces=all,
        sommets=
		0 0 2
		-1.5 -0.1 0
		1 -1 0
		1.5 2.5 0,
   faces={
       [0 1 2]
       [0 2 3]
       [0 3 1]
       [1 3 2]}](0,4.5,-4)%
\psSolid[object=new,
        action=draw**,
		fillcolor=cyan,
		RotX=120,
		numfaces=all,
        sommets=
		0 0 2
		-1.5 -0.1 0
		1 -1 0
		1.5 2.5 0,
   faces={
       [0 1 2]
       [0 2 3]
       [0 3 1]
       [1 3 2]}](0,9,-4)%
	   %%%%%%%%%%%%%%%% Y
\psSolid[object=new,
        action=draw**,
		fillcolor=cyan,
		numfaces=all,
        sommets=
		0 0 2
		-1.5 -0.1 0
		1 -1 0
		1.5 2.5 0,
   faces={
       [0 1 2]
       [0 2 3]
       [0 3 1]
       [1 3 2]}](0,0,-8)%
\psSolid[object=new,
        action=draw**,
		fillcolor=cyan,
		RotY=85,
		numfaces=all,
        sommets=
		0 0 2
		-1.5 -0.1 0
		1 -1 0
		1.5 2.5 0,
   faces={
       [0 1 2]
       [0 2 3]
       [0 3 1]
       [1 3 2]}](0,4.5,-8)%
\psSolid[object=new,
        action=draw**,
		fillcolor=cyan,
		RotY=125,
		numfaces=all,
        sommets=
		0 0 2
		-1.5 -0.1 0
		1 -1 0
		1.5 2.5 0,
   faces={
       [0 1 2]
       [0 2 3]
       [0 3 1]
       [1 3 2]}](0,9,-8)%
\end{pspicture}
\label{fig:Unique}
\caption{Unique}
\end{figure}

So a polyhedron is xy-axis will be unique trough rotation if there
exists symmetry in the polyhedron.
\newpage
\subsection{Give an algorithm that determines an $xy$-silhouette of a convex polyhedron $P$ in O(n) time}

\subsubsection{initialization}
To initialize we use a function that receives the first vertex as
described in the assignment.

\begin{algorithmic}[1]
	\Function{findxySilAlgorithm}{$v_{start}$}
	\State
	$\vec{e}_{init} \gets IncidentEdge(\Call{findxyBorderFirstImprovment}{v_{start}})$
	\State
	\Comment{So now we have an edge coming out of an origin with a
	maximized x value, however we still have to find the edge that
	goes over the silhouette}
	\State
	$\vec{e}_{init} \gets \Call{findFirstSilhouteEdge}{
		\vec{e}_{init}, Next(Twin(\vec{e}_{init})), \vec{e}_{init}
	}$
	\State
	\State \Return
	\Call{findxySilhouette}{
		$Origin(\vec{e}_{init}), \vec{e}_{init}, Seq(Origin(\vec{e}_{init}))$
	}
	\EndFunction
	\State
	\Comment{Finds the xy-border by maximizing X}
	\Function{findxyBorderFirstImprovement}{$v_{start}$}
		\State
		\Comment{The incident is used to check when we've traversed all edges}
		\State
		$\vec{e}_{incident} \gets IncidentEdge(v_{start})$
		\Repeat
			\State
			$\vec{e}_{next} \gets Next(Twin(\vec{e}_{incident}))$
			\State
			$v_{candidate} \gets Origin(Next(\vec{e}_{next}))$
			\If{$v_{candidate}$ has bigger x than $v_{start}$}
				\State
				\Comment{The return statement will break the loop}
				\State
				\Return \Call{findxyBorderFirstImprovement}{$v_{candidate}$}
			\EndIf
		\Until{$\vec{e}_{next}$ is equal to $\vec{e}_{incident}$}
		\State
		\Comment{If the loop finds no improvement we've maximized X and are on the border}
		\State \Return $v_{start}$
	\EndFunction
	\Function{findFirstSilhouetteEdge}{$\vec{e}_{start}, \vec{e}_{current}, \vec{e}_{best}$}
	\If{$\vec{e}_{current}$ is equal to $\vec{e}_{start}$}
		\State
		\Return $\vec{e}_{best}$
	\EndIf
	\If{$\vec{e}_{current}$ angle compared to the x axis is more shallow than $\vec{e}_{best}$}
		\State
		$\vec{e}_{best} \gets \vec{e}_{current}$
	\EndIf
	\State
	\Return \Call{findFirstSilhouetteEdge}{$\vec{e}_{start}, Next(Twin(\vec{e}_{current})), \vec{e}_{best}$}
	\EndFunction
\end{algorithmic}

\newpage
\subsubsection{Algorithm}
Ones initialization is done and we have the first edge that is on the xy
silhouette the findxySilhouette function is called.
This is the heart of the algorithm.
\begin{algorithmic}[1]
	\Function{findxySilhouette}{$v_{start}, \vec{e}_{current}, S_{result}$}
	\State
	\Comment{Base case, passing the start all the way down doesn't require us to do marking}
	\If{$v_{start}$ is equal to $Origin(\vec{e}_{current})$}
		\State \Return $S_{result}$
	\EndIf
	\State
	\Comment{Figuring out which edge to do next}
	\State
	$\vec{e}_{next} \gets Next(\vec{e}_{current})$
	\State
	\Comment{We use another recursive function for this}
	\State
	$\vec{e}_{next} \gets \Call{findLeastRotatingEdge}{Twin(\vec{e}_{current}), Next(Twin(\vec{e}_{next})), \vec{e}_{next}}$
	\State
	\Return \Call{findxySilhouette}{
		$v_{start}, \vec{e}_{next}, S_{result} \cup \vec{e}_{current}$
	}
	\EndFunction
	\Function{findLeastRotatingEdge}{$\vec{e}_{start}, \vec{e}_{current}, \vec{e}_{best}$}

	\Comment{Trying to get an as shallow rotation as possible}
	\If{$\vec{e}_{current}$ rotation is bigger than $\vec{e}_{best}$ (with as orientation point $Prev(\vec{e}_{start})$)}
		\State
		$\vec{e}_{best} \gets \vec{e}_{current}$
	\EndIf
	\State
	\Comment{Base case for recursion}
	\If{$\vec{e}_{start}$ is equal to $\vec{e}_{current}$}
		\State
		\Return $\vec{e}_{best}$
	\EndIf
	\Comment{Going deeper because there is another step}
	\State
	\Return \Call{findLeastRotatingEdge}{$\vec{e}_{start},Next(Twin(\vec{e}_{current})),\vec{e}_{best}$}
	\EndFunction
\end{algorithmic}

Because the Polyhedron $P$ is convex  and
the initial vertex $v_{start}$ was already on the boundary this
algorithm is correct.

\subsection{Why takes this algorithm linear time}
To determine this lets go over the function and see how much time they each
maximally can take.
To start the algorithm you call findxySilAlgorithm and pass it a starting
vertex. Then to find a boundary of the xy-silhouette the
FindBorderFirstImprovement is used. This tries to maximize X to ensure we
are on the border of the silhouette. It takes any improvement, so it may
not take the shortest path, but at most it will take $O(N)$ time.

After the border is found we pass it to the findxySilhouette function.
This function will take at most $O(N)$ time, this function will always
tries to stay on the least xy rotation as possible.

This function traverses all connecting edges of the current edge's endpoint,
so it sounds like we're having an $O(N^2)$ function here, but this isn't the
case because the outer function doesn't consider all elements
(only the outer ring) and the second function only considers the local
connections of the current edge. Because this is a local consideration
and most of these edges don't overlap (in fact only 2 do per consideration,
the starting on and the answer), it takes $O(N)$ running time.

So the total running time of this algorithm is:

\[ O(N+N) \]

Which simplifies to:

\[ O(N)\]

\section{Red and blue lines}
\subsection{algorithm}
To design the algorithm I wrote it down in pseudo code:
\begin{algorithmic}[1]
	\State
	$P_{ends} \gets R \cup B$ (flat) Mapped to endpoints where each segment
	produces two end points, the endpoints contain a reference to each other
	and they also contain which color they are and are also marked
	valid.
	\State
	$P_{sorted} \gets P_{ends}$ sorted on Y coordinate.
	\State
	$S_{tatus} \gets $ empty binary tree that sorts on X axis, it also
	returns the two direct neighbours on insert.
	\State
	$S_{result} \gets $ empty list
	\For{$p_{current} \gets Next(P_{sorted})$}
		\If{$p_{current}$ is a start point of a blue line}
			Insert $p_{current}$ into $S_{tatus}$
		\State
		\ElsIf{$p_{current}$ is a start point of a red line}
			\State
			$t_{neigbours} \gets $ Insert $p_{current}$ into $S_{tatus}$

			\If{$t_{neigbours}$ are not both blue lines}
				\State
				$p_{current}$ gets marked invalid.
			\EndIf
		\State
		\ElsIf{$p_{current}$ is an end point of a blue line}
			\State
			$t_{neigbours} \gets $ Insert $p_{current}$ into
			$S_{tatus}$
			\If{$t_{neigbours}$ a red line}
				\State
				mark that line invalid if $t_{neigbours}$ does
				not contain a blue line.
			\EndIf
			\State
			Remove $p_{current}$ and its twin (so the start point) from$S_{tatus}$
		\State
		\ElsIf{$p_{current}$ is an end point of a red line}
			\State
			$t_{neigbours} \gets $ Insert $p_{current}$ into
			$S_{tatus}$
			\If{both $t_{neigbours}$ are blue and the twin of $p_{current}$ was still marked valid}
				Add $p_{current}$ to $S_{result}$
			\EndIf
			\State
			Remove $p_{current}$ and its twin (so the start point) from$S_{tatus}$
		\EndIf
	\EndFor
	\State
	\Return $S_{result}$
\end{algorithmic}

\subsection{Status}
The status of the sweep line is a set of line segments intersecting it at
its current position. ordered from left to right (x axis).

The status is a binary tree with a special insert operation that also
returns the neighbours of the inserted element.

\subsubsection{Structure}
A node in the binary tree has a reference to a parent, and a left and right child.
A node has a value, which will be used to decide if you have to go left or right.
The initials node value is just the x coordinate.

\subsubsection{Insertion}
If another element gets
inserted we compare its x coordinate with that of the current node. If its
smaller we go left otherwise right. If x coordinate is equal we use the other
endpoint's X coordinate as a tie breaker, if that is equal to we look at
Y coordinate.
In case we are at the bottom of the tree we create a new connection node
and take the average of the old bottom element and the inserted element.
The parent node gets updated to let the right child point at the new
connection node and the new connection node has as children the old bottom
element and the newly inserted element.

\paragraph{Neighbours on insert}
But we defined a special binary tree that will return the neighbours upon
insertion. So at this point we already found one of the neighbours, to find
the other one we have to go up again until we can go into the neighbour
direction and this direction is not equal to where we cam from and then go
all the way in the other direction.
So if we try to find the right neighbour we go up and try to go to the right
child, if this is the child where we came from we go up further and try again.
Until we find the root, which means there is no right neighbour, or until
we find a child that is not the direction where we came from, which means
we have to go left all the way down in the tree.

\subsubsection{Removal}
Removal is done by finding the node we want to remove, then going to its
parent parent and saying one the child you came from is equal to its neighbour

\subsubsection{Balancing}
To ensure $O(log N)$ access time an AVL implementation
can be used. But I think describing that is somewhat out of the scope
of this assignment.

\subsection{Events}
The events are the beginning of either a red or blue line and the ending of
a red or blue line.

If the beginning of a blue line is encountered it gets inserted into the
status structure.

If the beginning of a red line is encountered it gets also
inserted into the status structure, Once inserted we check if the line left
and right are blue lines, if not we mark it invalid.

If the ending of a blue line is encountered we check its neighbours. If either
of them are red we make sure the other neighbour is blue, if not we mark the red line as invalid.
Then we remove the blue line from the status structure.

If the ending of a red line is encountered we insert the ending into the status structure.
If this ending has two blue lines as neighbours and its twin point wasn't marked invalid.
to the result set.
Finally we remove it and continue.

\subsection{Running time}
To do the plane sweep we first sort the points on Y access. This means
we have at least already a running time of:

\[ O((n+m) log (n+m))\]

Now the question is can we stay below this for the rest of the algorithm?
For the algorithm we consider all events which can be expressed as:

\[ 2(n+m) \]

Because for every red $n$ line there are two events, namely the begin point and
end point of that line.
So we loop trough this and use a binary tree to determine if there are neigbours
or not.
let $k$ be the size of $S_{tatus}$.
Doing the test for neighbours takes at most:

\[2 log(k)\]

Assuming its a balanced tree. Its times 2 because one neigbour is found for
free on insertion, but for the other neighbour in worst case we have to
consider the entire tree.

So the final running time for the algorithm is:

\[O(2(n+m) 2log(k))\]

Plus the initialization:

\[O((n+m)log(n+m)+2(n+m) 2log(k))\]

\section{Polygon cut}
Consider the following polygon:

\begin{tikzpicture}
	\draw (0,1.25) -- (2,1.25) -- (2,0) --
	 (1,1) -- (0,0) -- cycle;
\end{tikzpicture}

The best horizontal cut in this case is a cut that looks like so:

\begin{tikzpicture}
	\draw (0,1.25) -- (2,1.25) -- (2,0) --
	 (1,1) -- (0,0) -- cycle;
	 \draw [dotted] (0,1) -- (1,1);
\end{tikzpicture}

The reason for this is that the two triangles are bigger in area than
the connecting rectangle. But if we try to cut the entire figure we
end up with 3 different areas instead of 2. So the best cut we can
make will end up to  the depth of the triangle enclave.

Now if you make the rectangle bigger something happens:

\begin{tikzpicture}
	\draw (0,2) -- (2,2) -- (2,0) --
	 (1,1) -- (0,0) -- cycle;
	 \draw [dotted] (0,1.25) -- (2,1.25);
\end{tikzpicture}

The best cut moves up to divide everything perfectly.
So we get an imperfect cut that *has* to end on a vertex according to
the lemma. Or we get a perfect cut that may or may not end on a vertex.

\paragraph{To proof} Given a simple polygon $P$, the best horizontal cut
has at least one endpoint on a vertex of $P$, or it cuts $P$ into two pieces
with equal area (or both).

\paragraph{Proof} There are two cases to be proven:
	In case we can best horizontal cut a polygon $P$ into two equal areas
	we're done (since in this case there are no restrictions on the best
	horizontal cut line).

	In case we can't best horizontal cut polygon $P$ into two equal areas,
	the best horizontal cut has to minimise the larger area. In this case
	the best horizontal cut has to end on a vertex of $P$ according to
	the lemma.

	To prove this lets assume its not the case a best horizontal cut's endpoint
	ends lies on a vertex of $P$ if dividing into two equal areas is impossible.
	However because the best horizontal cut has to minimize the larger
	area it will always end up on some sort of angle of the polygon, which
	is defined by a vertex. It can't equally divide the polygon because
	that was already handled by case one.
	So it can't be the case that the best horizontal cut does not
	end up on a vertex of $P$. This is what had to be shown.

\subsection{Vertex types}
Merge and split by definition, but on occasion a regular may happen to.
\section{Diagonal polygon cut}
The most obvious solution is the brute force solution, which involves
having two loops that compare every possible cut from every vertex,
while keeping track of the best cut. This will take $O(n^2)$ time.

A faster method is just finding all the merge and split vertexes. Which takes
$O(n)$ time, and then comparing each of these with all other vertexes and keeping
track of the best area. This will take at most $O(n+(m+s)n)$
with $m$ as merge vertexes and $s$ as split vertexes. This is correct because in
exercise 3.b we determined that the vertexes a best horizontal cut intersects
with either a merge or split vertex, so the best diagonal cut will to.
\end{document}
