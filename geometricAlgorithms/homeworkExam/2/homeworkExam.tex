\documentclass{article}
\usepackage{amsmath}
\usepackage{algorithmicx}
\usepackage{algpseudocode}
\usepackage{pst-solides3d}
\usepackage{auto-pst-pdf}
\usepackage{tikz}
\usepackage{enumerate}
\renewcommand{\thesubsection}{\thesection.\alph{subsection}}
\begin{document}
\author{Jappie Klooster}
\title{Stay in shape! Homework exam II}
\maketitle
\section{A disk and some half-planes}
\subsection{Per case, what can you say about $H$}
We number the half-planes of $H$: $h_1, h_2,\dots,h_i$.
Let $H_{i}$ be the set of the first $i$ constraints, and
let $C_{i}$ be the feasible region defined by these constranits:

\[ H_i := \{h_1, h_2,\dots, h_i\}\]
\[ C_i := h_1 \cap h_2 \cap \dots \cap h_i\]

\begin{enumerate}[(i)]
	\item \emph{There is no disk that has its centor non the $x$ axis
			that lies inside all half-planes.}

		$C_i$ never crosses the x axis,
		so it exists entirely above or below $y=0$. Or the intersection is empty.

	\item \emph{There are arbitrarly large disk with the specifications.}

		$C_i$ has no parallel line  to te x axis and
		at most one line crossing the x axis.

	\item \emph{The largest disk exists and is unique.}

		$C_i$ crosses the x-axis at least 2 times and
		if there is a line parallel to the x axis then both x-axis crossing
		to be in range of the diameter of the disk.

	\item \emph{There are several equally large largest disks}

		$C_i$ produces an area with lines parallel to the x-axis.

\end{enumerate}

\newpage
\subsection{Deal with case ii in linear worstcase time}
\begin{algorithmic}[1]
	\Function{hasArbitrarlylargeDisc}{$H$}
		\State
		$b_r, b_l \gets None$
		\For{$i \gets 1$ to $n$ with $h_i \in H$}

			\If{$h_i$ is parallel to the x-axis}
				\State
				\Return $False$
			\EndIf
			\State
			\Comment{Check where the x axes crossings are, and write it into
			a variable pair, if both pairs are filled its not case ii.}
			\If{$h_i$ points left}
			\State
				$b_r \gets h_i$
			\EndIf
			\If{$h_i$ points right}
			\State
				$b_l \gets h_i$
			\EndIf
			\If{$b_r$ not equal to $None$ and $b_l$ not equal to $None$}
				\State
				\Return $False$
			\EndIf
		\EndFor
		\State
		\Return $True$
	\EndFunction
\end{algorithmic}
\subsection{Formulate and proof the RIC lemma}
\paragraph{Lemma}
Let $1 \leq i \leq n$, let $C_i$ be the defined as above and let $d_i$ be
the largest disk in $C_i$. Then we have
\begin{enumerate}[(i)]
	\item If $d_{i-1} \in h_i,$ $then$ $d_i = d_{i-1}$
%TODO: C = nill? dat klopt volgens mij niet
	\item If $d_{i-1} \not\in h_{i},$ $then$ $C_i = \emptyset$
		or $b_i \in l_i$ where $b_i$ is the boundary of $d_i$ and $l_i$
		is the line bounding $h_i$.
\end{enumerate}
\paragraph{Proof}
\begin{enumerate}[(i)]
	\item Let $d_{i-1} \in h_i$. Because $C_i = C_{i-1}\cap h_i$ and
		$d_{i-1} \in C_{i-1}$. Furtheremore,the optimal disk in $C_i$
		cannot be better than the optimal disk in $C_{i-1}$, since
		$C_i \subseteq C_{i-1}$. Hence, $d_{i-1}$ is the optimal disk
		in $C_i$ as well.
	\item Let $d_{i-1} \not\in h_{i}$. Suppose for a contradiction that
		$C_i \not\emptyset$ and that $b_i \not\in l_i$. Now consider
		the line segment $\overline{cx_{i-1}cx_i}$ where $cx_i$ is the
		center on the x-axis of disk $d_i$. We have
		$d_{i-1} \in C_{i-1}$ and, since $C_i \subset C_{i-1}$, also
		$d_i \in C_{i-1}$. 
\end{enumerate}

\subsection{Give the RIC algorithm}

\subsection{Prove the expected running time}
% Don't forget to remove constants
\subsection{Eplain why you can't use RIC for points}
\section{A datastructure for horizontal line segments}

\subsection{Analyze the storage requirements}
%TODO: warning: this is x and y descisions.
\subsection{Argue that all horizontal line segs are found}
\subsection{Give psuedo code for the query algorithm}
%TODO: sla lijnen op geen punten.

\subsection{Analyze the query time}

\section{Voronoi nearest neigbour\emph{s}}
%TODO: wtf? maybe with higher order voronoi?? But it won't be log n
%TODO: we hebben ook bogen. Kun je representateren als een lijn en een punt.
\end{document}
