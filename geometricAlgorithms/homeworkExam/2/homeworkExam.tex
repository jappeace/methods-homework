\documentclass{article}
\usepackage{amsmath}
\usepackage{algorithmicx}
\usepackage{algpseudocode}
\usepackage{pst-solides3d}
\usepackage{auto-pst-pdf}
\usepackage{tikz}
\usepackage{enumerate}
\renewcommand{\thesubsection}{\thesection.\alph{subsection}}
\begin{document}
\author{Jappie Klooster}
\title{Stay in shape! Homework exam II}
\maketitle
\section{A disk and some half-planes}
\subsection{Per case, what can you say about $H$}
\begin{enumerate}[i]
	\item \emph{There is no disk that has its centor non the $x$ axis
			that lies inside all half-planes.}

		The intersection of halfplanes of $H$ never crosses the x axis,
		so it exists entirely above or below $y=0$.

	\item \emph{There are arbitrarly large disk with the specifications.}

		The intersection of halfplanes of $H$ do not enclose the disk
		area.
		In fact the area they do enclose is smaller than $\pi$ radials.

	\item \emph{The largest disk exists and is unique.}

		The intersection of halfplanes of $H$ produces an area with no
		parralel lines.

	\item \emph{There are several equally large largest disks}

		The intersection of halfplanes of $H$ produces an area with
		parralel lines.

\end{enumerate}

\subsection{Deal with case ii in linear worstcase time}
throw new exception(``done'');
\subsection{Formulate and proof the RIC lemma}
\subsection{Give the RIC algorithm}
\subsection{Prove the expected running time}
% Don't forget to remove constants
\subsection{Eplain why you can't use RIC for points}
\section{A datastructure for horizontal line segments}
\subsection{Analyze the storage requirements}
\subsection{Argue that all horizontal line segs are found}
\subsection{Give psuedo code for the query algorithm}
\subsection{Analyze the query time}
\section{Voronoi nearest neigbour\emph{s}}
%TODO: wtf? maybe with higher order voronoi?? But it won't be log n
\end{document}
