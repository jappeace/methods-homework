\documentclass{article}
\usepackage{amsmath}
\usepackage{txfonts}
\usepackage{booktabs}
\usepackage{color}
\usepackage{bussproofs}
\usepackage[pdftex]{graphicx}
\DeclareGraphicsRule{*}{mps}{*}{}
\usepackage{pifont}
\usepackage{qtree}
\usepackage{emp}
\usepackage{listings}
\usepackage{hyperref}
\usepackage{svg}
\usepackage{pgfplots}
\usepackage{pgfplotstable}
\usepackage{seqsplit}
\usepackage{algorithmicx}
\usepackage{algpseudocode}
\pgfplotsset{
    compat=1.9,
    compat/bar nodes=1.8
}

\newenvironment{scprooftree}[1]%
{\gdef\scalefactor{#1}\begin{center}\proofSkipAmount \leavevmode}%
{\scalebox{\scalefactor}{\DisplayProof}\proofSkipAmount \end{center} }


\newcommand{\brcell}[2][l]{%
	\begin{tabular}[#1]{@{}l@{}}#2\end{tabular}}

% "define" Scala
\lstdefinelanguage{scala}{
  morekeywords={abstract,case,catch,class,def,%
    do,else,extends,false,final,finally,%
    for,if,implicit,import,match,mixin,%
    new,null,object,override,package,%
    private,protected,requires,return,sealed,%
    super,this,throw,trait,true,try,%
    type,val,var,while,with,yield},
  otherkeywords={=>,<-,<\%,<:,>:,\#,@},
  sensitive=true,
  morecomment=[l]{//},
  morecomment=[n]{/*}{*/},
  morestring=[b]",
  morestring=[b]',
  morestring=[b]"""
}
\begin{document}
\begin{empfile}
\begin{empcmds}
input metauml;
\end{empcmds}
\author{Jappie Klooster}
\title{Examination of shapes! (homework exam)}
\maketitle

\section{Polyhedron}

\subsection{When is the $xy$-silhouette of a convex polyhedron $P$ is unique}
%TODO this anwser seems to complex for 0.5 points, maybe I should look in the book
% TODO: don't give a crappy discription
%TODO: define uniqueness?
First we have to define what uniqueness means to anwser this question. I'm 
assuming uniqueness in this case means an $xy$-silhouette that can only be
generated from a polyhedron P when looking at it from a specific angle.

So when projecting a polyhedron P on an $xy$ plane (ie removing the z values).
you need to have 3 angles. $x$ rotation, $y$ rotation and $z$ rotation.

An $xy$-silhoutte is said to be unique if there is only one set of rotations
that can generate that sillhouette.

I had initially some questions about this defintion, because is it not possible
to always generate another sillouthe by just applying enough rotations.
So I started drawing a simple sillhoutte, a square with an angle.

\begin{figure}
\begin{tikzpicture}[
rotor/.style={inner sep=3pt,outer sep=0,fill opacity=0.3,minimum width=2cm,circle},
crossline/.style={gray,line width=3pt,opacity=0.5,line cap=round}
]
	\node[rotor] (n2) at (4,2)  {\quad y};

	\draw [->,line width=1pt] (n2) ++(140:5mm) arc (-220:40:5mm);
	\draw (1,1) -- (3,1) -- (3,3) -- (1,3) -- (1,1);
	\draw (2,2) -- (0,2) -- (0,1);

	\draw (5,1) -- (7,1) -- (7,3) -- (5,3) -- (5,1);
	\draw (6,2) -- (8,2) -- (8,1);

	\draw (4,0) -- (4,4);
\end{tikzpicture}
\caption{Cube with arm sticking out sillhoutte}
\end{figure}

I did applied similiar rotations for the $x$ and $z$ axis. If you
do both 180 degrees you end up looking at the same side. So for this
particiluar figure there exists a unique $xy$-sillhoutte.

\subsubsection{When it is not unique}

It is not unique when certain parts of the figure can be mirrored. For
example consider the follwoing figure:

\begin{figure}
\begin{tikzpicture}[
rotor/.style={inner sep=3pt,outer sep=0,fill opacity=0.3,minimum width=2cm,circle},
crossline/.style={gray,line width=3pt,opacity=0.5,line cap=round}
]
	\node[rotor] (n2) at (5,2)  {\quad y};

	\draw [->,line width=1pt] (n2) ++(140:5mm) arc (-220:40:5mm);
	\draw (1,1) -- (3,1) -- (3,3) -- (1,3) -- (1,1);
	\draw (2,2) -- (0,2) -- (0,1);
	\draw (2,2) -- (4,2) -- (4,3);

	\draw (7,1) -- (9,1) -- (9,3) -- (7,3) -- (7,1);
	\draw (8,2) -- (10,2) -- (10,1);
	\draw (8,2) -- (6,2) -- (6,3);

	\draw (5,0) -- (5,4);
\end{tikzpicture}
\caption{Cube with two arms in opposite direction sillhoutte}
\end{figure}

The issue with this figue is that if you apply 180 degrees $z$ rotation you
end up with the same sillhoutte. So having some symetry ends up causing will
cause having redundant sillhouttes.

\subsection{Give an algorithm that detrmines an $xy$-silhouette of a conves polyhedron $P$ in O(n) time}
%TODO: should we find the side?

\begin{algorithmic}
	\Function{findxySilhouette}{$v_{start}, \vec{e}_{current}, i_{rotation}, S_{result}$}
	\If{$v_{start}$ is equal to $Origin(\vec{e}_{current})$}
		\State \Return $S_{result}$
	\EndIf
	\State
	$\vec{e}_{next} \gets Next(\vec{e}_{current})$

	\State
	$i_{currentRot} \gets i_{rotation} + $ rotation between 
	$\vec{e}_{current}$ and $\vec{e}_{next}$, rotation they make on the 
	face (in radians).
	\If{$i_{currentRot}$ bigger then $ 0.5\pi$}
		\State \Return \Call{findxySilhoeutte}{
		$v_{start}, Next(Twin(\vec{e}_{next})), 0, S_{result} \cup \vec{e}_{current}$
	}
	\EndIf
	\State
	\Return \Call{findxySilhoeutte}{
		$v_{start}, \vec{e}_{next}, i_{currentRot}, S_{result} \cup \vec{e}_{current}$
	}
	\EndFunction
	\State
	\Comment{Helper function to suffice the contract (init work)}
	\Function{findxySilAlgoirthm}{$v_{start}$}
	\State $\vec{e}_{init} \gets IncidentEdge(v_{start})$
	\State
	\Comment{The $Seq$ just makes an arbitrary sequence with constand append time}
	\State \Return
	\Call{findxySilhoeutte}{
		$v_{start}, \vec{e}_{init}, 0, Seq(\vec{e}_{init})$
	}
	\EndFunction
	\State 
\end{algorithmic}

Because the Polyhedron $P$ is convex (so no weird dents), and
the initial vertex $v_{start}$ was already on the boundry this
alogrithm is correct.
In class we used marker bits but its simpler to just pass the
starting point, less chance to shoot yourself in the foot with
mutable state.

\subsection{Why takes this algorithm linear time}
Because all operations are linear or faster. In fact the slowest
operation, the recursive step is in itself linear. It is gauranteed to stop
as long as it is passed a cycle.

\section{Red and blue lines}
\subsection{The status structure}
A binary tree sorted on the x axis.
\subsection{Initialization}

\subsection{Distinguish the events and describe for each event how its handled}
\subsection{Analyze the time}
\section{Polygon cut}
\section{Diagonal polygon cut}
\end{empfile}
\end{document}
