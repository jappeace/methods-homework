\documentclass{article}
\usepackage{booktabs}
\usepackage{amsmath}
\usepackage{color}

\begin{document}
\author{J. T. Klooster}
\title{Homework of 14 september}
\maketitle
\today
\section{Prove that Hilbert system $H$ is sound with respect to this
semantics: If $\vdash_H A$, then $A$ is true.}
%TODO: I have no idea

From the slides the thoerom says that $H$ is sound and complete
if and only f $A$ is a tautology.

Proof $A$ in $H$ is a sequance $B_1, B_2, \ldots, B_n$

\begin{tabular}{@{}l|ll@{}}
\#& statement			&reason \\ \toprule
0 & Formula $A$ is true if $(v(A) = 1$ & given
1 & If $\vdash_H A$ then $A$ is true & given
\end{tabular}


%Where you are asked to show that if a formula A is derivable in system H,
%then A is true (under all valuations).
%This is a statement that you can best prove via induction to the length of
%the proof of A. Recall that a proof of A in H is a sequence B1,B2,...,Bn,
%of formulas such that such and so holds (see slides of today).
%Then you show that the statement holds in case n=1
%(when the proof consists of one formula) and that if the statement holds
%for formulas with proofs of length n,
%then it holds for formulas with proofs of length (n+1), the induction step.
\section{Prove that $\vdash_H A \to B$ and $\vdash_H A$, then $\vdash_H$ B}
To prove this $A \to B$ has to transformed with the hilbert system to $B$

\begin{tabular}{@{}l|ll@{}}
\#& statement			&reason \\ \toprule
0& $\vdash_H A$ & given \\
1& $\vdash_H A \to B$ & given \\
2& $B$ & Modus Ponens 0,1 \\
\end{tabular}

\section{Prove that it is not the case that $\vdash_H A \to B$ implies 
$ \vdash_H \neg A \to \neg B$}

\begin{enumerate}
	\item $A \to B$ implies $\neg A \to \neg B$ can be written as 
		$(A \to B) \to (\neg A \to \neg B)$
	\item If A was $\bot$ and B was $\top$ then $(A \to B) \to (\neg A \to 
		\neg B)$ would be false (see truth table below), therefore it is not a tautology and cannot
		be used in a hilbert system.
\end{enumerate}
\subsection{truth table}
\begin{tabular}{@{ }c@{ }@{ }c | c@{}@{}c@{}@{ }c@{ }@{ }c@{ }@{ }c@{ }@{}c@{}@{ }c@{ }@{}c@{}@{ }c@{ }@{ }c@{ }@{ }c@{ }@{ }c@{ }@{ }c@{ }@{}c@{}@{}c@{ }}
A & B & ( & ( & A & $\rightarrow$ & B & ) & $\rightarrow$ & ( & $\sim$ & A & $\rightarrow$ & $\sim$ & B & ) & )\\
\hline 
$\top$ & $\top$ &  &  & $\top$ & $\top$ & $\top$ &  & \textcolor{red}{$\top$} &  & $\bot$ & $\top$ & $\top$ & $\bot$ & $\top$ &  & \\
$\top$ & $\bot$ &  &  & $\top$ & $\bot$ & $\bot$ &  & \textcolor{red}{$\top$} &  & $\bot$ & $\top$ & $\top$ & $\top$ & $\bot$ &  & \\
$\bot$ & $\top$ &  &  & $\bot$ & $\top$ & $\top$ &  & \textcolor{red}{$\bot$} &  & $\top$ & $\bot$ & $\bot$ & $\bot$ & $\top$ &  & \\
$\bot$ & $\bot$ &  &  & $\bot$ & $\top$ & $\bot$ &  & \textcolor{red}{$\top$} &  & $\top$ & $\bot$ & $\top$ & $\top$ & $\bot$ &  & \\
\end{tabular}

\section{Prove that $H$ and $G$ are equivalent}
Let $G$ be the proof system $H$ to which the axiom $A \to A$ 
is aded and in which axiom $(W)$ is replaced by the \emph{weakining rule}

\[\frac{A}{B \to A}\quad\text{(RW)} \]

I would say if you could transform G to have the same
axioms as H then you would have proved that they're equivelant. But I think
you also have to show that you can go from G to H because equivulance goes both
ways

\end{document}
