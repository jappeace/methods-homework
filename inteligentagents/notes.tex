
#Exam

You can use all material except that about eleborated excersizes.
You should be able to work with it. You can mark stuff.

#known
Stuff in mehthods of AI should be known
Modal logic is very important.

meyer & van der hoek, epistamic logic for AI and computer science, cambridge
univ press.
Very usefull for self study. Because it has excersizes with eleborations in
the back.

#Inteligent agents

Has to do with behaviour, try to mimic bahavior from the natural world in
software.

Agents, has many definitions but in this course its something with a
coginitive state.
Believe desires intentions - in a methorphorical stance, so not really

## Complete autonomy
Is kindoff useless, we need limited autonomy.
We need some kind of initiative or proactive or goal-directed.

##Mostly described in having mental states.
BDI 'strong notion of agency'

Show informational and motivational attitudes. an agent has a
motivation system of its own. It can be programmed, but then
the agent is on its own.

#Properties of agents.

##Situated
It can sense and reason about the environment.

They don't just give advice like exper systems

##reactive

## proactive

persuin goals, make plans.


## social
compete with eachorther.

#Objects vs agents

objects have some autonomy, but agents its more involved.

Agents are more 'subject' than 'object' in linguistic

objects can just be used, can be wrong but ok. An agent however
can only ask, so it can be refused. This isn't failing
its just the way it works.


... agents are objects with an attitude... wil of its own.
This is just an idea and there is an entire course about how we can do this.
It should be logical.

#Multi agent systems
After christmess.

There are quite difficult logics to describe their behavior.
You want multiple agents to wrok on a complex task, so you want to devide
the labour. if one fails others can take over.
Gracefull degradation. <- if somethign is wrong not everything is over

## why MAS

also paralism
roles of which a task can be done

## application

Softbots

### Robots
The rottordam harbour, the cranes are agent oriented. work in progress though.
Self driving cars, that could comminucate....

Replace the controller by an agent. Cognotive robotics, robots that think.
A robot steared by an agent.

Itel?? agents/robots exchanging goals and believes.

### Industrial
Airtraffic, one of the first applications of agents.
Manufactiring more agile, which can change easily

### Comericial
Automatic auctions. specifiy limits and the agent will negotiate.

### Entertainment
Virtual characters, more intelligent. Common sense.
Cinema <-- titanic was the first one that used agents.

# courses
##Philosophy
sloman: <- emotional robots


###Artitsotle
Practical sylogism, reasoning about actions.

### Denet
Strong AI, the computer can implement thinking like a human, as complex the
human brain.

###BRatman intetnions
Something to do with desires. But he thought about how to implement it with
software.

##Logical foundations
It needs to be combined. Quite complicated.

##Reasoning about action and frame
Frame problem, implementation of actions does not change much, but the rest does.

Situation calculus, <- to controll robots

##Agent architectures
#Bratman

Commitment, if you have a real intention, you want to go for it.
intentions should be consistent, otherwise you do something which could be
impossible
But this is debatable...


## Intentions
Screen of admissibility, filter maybe later intentions that could be
conflicitng


# Cohen& levesque
tried to formalise Bratman.

## Intention = choice + commitment
try to stretch it to formal logic.

##
agent beleives P is possible : a sort of self conficience of the agent

## side effects
agnets need not intend all the expected side effects of their intentions.
Going to the dentist implies pain.
then going to the dentist should be that you intend to have pain.

$\vDash \phi \to \psi \to \vDash \Box \phi \to \Box \psi$

#Rao & Georgeff BDI theory
Not only proposed a logic, but also the first implementation of an agent and
also the first application.

#Rodney Brooks anti-cogn.robotics
subsumption architecture: no cognition, just a reactive system.
Should be emergent behaviour rather than modeling.

# Aaron Sloman: How to build emotional agents
Looks at properties of humans, such as emotions, and tries to put it into an
AI.

20 years ago, there are patiens that don't have emotions, expect: in problem
solving such as puzzels. They go depth first, so they go on, and on and fail
misserably.

Archicterus of him are not the most enlighted, "messy".

# Towards social agents/robots
also research goes to sociology.

Oppertunisitic behavior, sellar buying a cup, but cup is broken,
the seller wants to sell it, but the buyer wants to use it. How
to moddel this in Agents

## Normative systems
the way of societies. you shouldn't violate the norm, but sometimes it can
be usefull to.

### Asimov
Laws of him. 4th one is stupid

# Theory of mind
The psycology one.
The false believe test.


# Logial foundations of actions

The terminology is not stabels.
desires and goals may be differetn or mean the same.

Agenda -> latin things that should happen.

# Modal logic
Logic of $\Box$ and $\Diamond$

## K axiom
true for any modal logic

## R
means its accesible. No restriction on R in arbitrary Kripke


### Constraints
you can add constraints on R, but with restrictions things certain
things become true.

serial you can go on.

## Frames
Models where the true assignment functions are abstracted from.

#naive BDI logic
seperate blocks, it misses connections.
No dymancis, changes of BDI in time.

# Cohen & levesque : langauge
$\alpha$ are complex actions and a's are primitive actions
We can't say multiple agents do the same action at the same time.

a is interpeted as a sequance of primitive events even though a is
atomic.

atomic action is a special class, which depends on the domain.

; = sequence
? = test if its true (groovy truth)
+ = is a non detrmenistic choice, collect them in an autcome set.
* = is done a finite but arbitrary number of times, including 0

if $\phi$ then $\alpha_1$ else
$\alpha_2$ = ($\phi? ; \alpha_1) + (\neg \phi?;\alpha)$

while $\phi$ do $\alpha$ = $(\phi?;\alpha)^*;\neg\phi?$

Pz = z has property P, a predicate attom


# Semantics
\omega = variables assginemnts
P = is the set of omega \in \omega
E = is a set of primitive events
Agt = Agt \in [E \to p] specifies the agent of an event [E \to P] is a
function that takes an E that gives a P.
T \subset [Z \to E]; a set of possible worlds, give a time number and get



\documentclass{article}
\begin{document}
\author{Jappie Klooster}
\title{Cohen \& Levesque}
\maketitle
\section{Langauge}
DONE = just done
; = sequence
? = test if its true (groovy truth)
+ = is a non detrmenistic choice, collect them in an autcome set.
* = is done a finite but arbitrary number of times, including 0

if $\phi$ then $\alpha_1$ else
$\alpha_2$ = ($\phi? ; \alpha_1) + (\neg \phi?;\alpha)$

while $\phi$ do $\alpha$ = $(\phi?;\alpha)^*;\neg\phi?$

Pz = z has property P, a predicate attom


\subsection{Semantics}
$\omega$ = variables assginemnts
$P$ = is the set of omega $\in \omega$
$E$ = is a set of primitive events
$Agt = Agt \in [E \to p]$ specifies the agent of an event $[E \to P]$ is a
function that takes an E that gives a P.
$T \subset [Z \to E];$ a set of possible worlds, give a time number and get

i = agent
n = time

\subsubsection{Predicate}
c constants
f functions
Q predicate smbols


$\phi$ interperts predicate symbols

\[\phi(c) = d \in \Theta\]
\[\phi(f) = f D_1x\dots D_l \to D_h\]
\[\phi(Q^{n,x,ne}) = R \subseteq D_1x\dots D_l \]

\subsubsection{satisfaction}

\[M,\sigma,v,n[\alpha] m\]
In the model `$M,\sigma,v,n$', $\alpha$ lies betweeen n and m.

B = `normal` modal accesibility relation indexed by the time $n$
and the agent $v(i)$. they write it like:

\[<\sigma, n>B[v(i)] \sigma\]

Better would be:
\[\sigma B_{v(i)}^n \sigma\]

\subsection{Constraints}
\subsubsection{Believes/Goals}
KD45.

\[G\subseteq B\]
means $\vdash {Bel}_i \phi \to {Goal}_i\phi$, but not the otherway around

\subsection{Achievement goal}
Doesn't believe $\phi$ is true yet, but believes it could be true later.
There exists some action to make $\phi$ true.

\subsubsection{Asumption}
Goals will end some time in the future

\subsection{Persistent goals}
Drop the goal if its true, or its impossible.

Competence means that if he believes it then its true.

\section{Rao Georgeff}
\subsection{Time logics}
Brancing vs linear, points vs intervals,
discrete (real numbers) vs continues (complex number)
We'll only look into the future.
\end{document}

\documentclass{article}
\begin{document}
\author{Jappie Klooster}
\title{Temporal magick or logic}
\maketitle
\section{Linear temporal magick}

$\phi_1\cup\phi_2$ = $\phi_1$ holds until $\phi_2$ becomes true.
$\Diamond\phi$ = sometimes
$O\phi$ = next time

$M=(S,\sigma, \pi)$ where $S$ is a set of states, $\sigma: N \to S$ is an
infinte seqence of states

\subsection{Semantics}
$\vdots$
$M,\sigma\vDash O\phi \LeftRightArrow M, \sigma^1 \vDash \phi$ next time
$M,\sigma\vDash \Diamond\phi \LeftRightArrow M, \sigma^n \vDash \phi$
invention, there is some time where $\phi$ is gonna hold.

\subsubsection{untill}
if $\phi_2$ holds somewhere it does not mean it holds everywhere after that
point

$\phi_1$ holds until $\phi_2$ holds.

weak untill = where $\phi_1$ holds everywhere because there is no
$\phi_2$, for example $\phi_1^{\bot}$

\section{Branching temproal magick}

Two kinds of formulas, state formulas, path formulas.
Any state formula can also act as a path formula.

\subsection{Models}
serial, any point has a continuation, time never stops.
$S \times S$ = cartesian product of S.

Fullpath is a ninfinite seq of states more or less true the tree.
These are more or less the lines of linear logic.
Fullpaths can start anywhere, not necisarly at the root.

$\sigma_1$ = already a choice of the branch, so no need for no trees.

\section{Combination of BDI}
E = optional
A = inevitable

\section{Operators magick}
$\Diamond\Diamond\phi=\Diamond\phi$
$\Box\Box\phi=\Box\phi$
$\Box\Diamond\phi$=Infenitly often there is a point where $\phi$ holds.
$\Diamond\Box\phi=$Some point in the future $\phi$ will hold indevinitly

\subsection{BDI}
Optionally $\Diamond p$ = there is a choice where p is gonna hold
Optionally $\Box r$= there is a timline where r is gonna hold
Inevitiably $\Diamond s$=  ???
Inevitiably $\Box b$=  b always holds

\section{BDI sementics}
s.t. = such that.

From t1 to t2, the agent has a memory, but he can doubt about the future.
There is the hidden asumption that agents have memory.

success = there exist some period in wehre the even was succesfull

\subsection{Axiom}
written like $GOAL(\alpha) \subset BEL(\alpha)$, but russel notation :s so
$\subset = \to$
$GOAL(\alpha) \to BEL(\alpha)$ = believe-goal compatibility
different from cohon and laveque, $BEL \alpha \to GOAL \alpha$, because
its a different framework, but the reason behind it is the same strangely
enough.

$INTEND(does(e))\to does(e)$ intention leads to action, which is controversial

\subsection{compatibility}
Sinning = ???, something with cutting of branches.

\section{CTD}
\subsection{Axioms}
$done(e) \to BEL(done(e))$,  you're awere that you've done something


\section{Praktical}
The general strategy is to write down what you have to proof and then to
read with help of the definitions.

Don't be sloppy and take an arbitrary M with for all's
\subsction{3.16}
\[\vDash \alpha \Rightarrow \vDash BEL x \alpha\]
Assume $\Vdash \alpha$, so $M,\sigma,v,n\vDash \alpha$ for all $M,\sigma,v,n$(*)
To prove $\vDash BEL x \alpha$, 
	ie $M,\sigma,v,n\vDash BEL x \alpha$ for all $M,\sigma,v,n\vDash$
	ie $M,\sigma,v,n\vDash\alpha$ for all $\sigma'$ with (\sigma,n)B[v(x)]
	\sigma' done by (*).

\subsection{3.26}
	To prove: $M,\sigma,v,n\vDash BEL x \alpha \to Goal x \alpha$ for all 
	$M,\sigma,v,n$.
	ie $M,\sigma,v,n\vDash BEL x \alpha \Rightarrow M,\sigma,v,n\vDash$ Goal
		$x \alpha$ for all $M,\sigma,v,n$
		ie $M,\sigma',v,n\vDash \alpha$ for all $\sigma'$ with $\sigma,n)
		B[v(x)]\sigam'\RIghtarrow M,\sigma'',v,n\vDash \alpha$ for all
		$\sigma''$ with $(\sigma,n)G[v(x)]\sigma''$

		Given:
		$(\sigma,n)G[v(x)]\sigma''\Rightarrow(\sigma,n)B[v(x)]\sigma''$
		$\vdots$

		Take arb $\sigma''$ with $(\sigma,n)G[v(s)]\sigam''$ then also 
		$(\sigma,n)B[v(x)]\sigma''$ so $M,\sigma,v,n\vDash \alpha$
\subsection(3.11)
Proof
\begin{prooftree}
	\AxiomC{}
	\UnaryinfC{}
\[ \vDash \neg LATER(\Diamond p) \neg LATER(\Diamond p) \LeftRightarrow 
\neg(\neg\Diamond p) \Leftrightarrow \neg(\neg\Diamond p \wedge 
\Diamond\Diamond p) \Leftrightarrow \neg(\neg\Diamond p \wedge 
\Diamond p)\]

\end{prooftree}

\subsection{3.13}
\begin{prooftree}
	\AxiomC{$\vDash \Diamond q \wedge BEFORE pq \to \Diamond p$}
	\UnaryInfC{To prove $M,\sigma,v,n\vDash \Diamond q \wedge BEFORE pq \to \Diamond p$ for all $M,\sigma,v,n$}
	\UnaryInfC{Take arbitrary $M,\sigma,v,n$ to prove $M,\sigma,v,n\vDash \vDash \Diamond q \wedge BEFORE pq \to \Diamond p$ for all $M,\sigma,v,n$}
	\Rightlabel{ie}
	\UnaryInfC{To prove if $M,\sigma,v,n\vDash \Diamond q$ and 
	$M,\sigma,v,n\vDash BEFORE pq$ then $M,\sigma,v,n\vDash Diamond p$ for all $M,\sigma,v,n$}
	\UnaryInfC{Supose $M,\sigma,v,n\vDash \vDash \Diamond q$ and 
	$M,\sigma,v,n\vDash BEFORE pq$}
	\Rightlabel{ie}
	\UnaryInfC{Supose $M,\sigma,v,n\vDash \exists x HAPPENS x;q?$ and 
		$M,\sigma,v,n\vDash \exists \forall c (HAPPENS c;q{?}) \to \exists q$ 
	$q\se c \wedge HAPPENS a;p{?}$}
	\Rightlabel{ie}
	\UnaryInfC{Supose $M,\sigma,v,n\vDash \exists x HAPPENS x;q?$ and 
		$M,\sigma,v,n\vDash \exists \forall c (HAPPENS c;q{?}) \to \exists q$ 
	$q\se c \wedge HAPPENS a;p{?}$}
\end{prooftree}

\end{document}
\documentclass{article}
\begin{document}
\author{Jappie Klooster}
\title{KARO logic}
\maketitle

There are indeed unkown unkowns. But not if you have
$\neg K \phi \to K \neg K \phi$

\section{the names are just the axioms}
jjK = ??
D = $\neg B \bot$
4 = $B\phi \to BB \phi$
5 = $\neg B \phi \to B\neg B \phi$

\section{Dynamic logics}
these actions can be non deterministic\ldots

Accessibility relation

$\alpha;\beta$ = sequentu\alphal compisition
$\alpha+\beta$ = nondeterministic choice
$\alpha*$ = execute \alphar\betaitr\alphary num\betaer of times

s' = a different variable to s, so instead of s' you can use [a-z].

Dynamic logic is complete!

does $[\alpha] ( \phi \to \varphi) \to ([\alpha] \phi \to [\alpha] \varphi)$
mean I can subtitute  $[\alpha] ( \phi \to \varphi)$ with
$([\alpha] \phi \to [\alpha] \varphi)$

$[\alpha*]\phi\to\phi$ do nothing its true
$[\alpha*]\phi\to[\alpha][\alpha*]\phi$ do once and it becomes true
\subsection{with induction}
$[\alpha*](\phi \to [\alpha]\phi)\to(\phi\to[\alpha*]\phi)$ 
$\phi\to[\alpha]\phi$
$[\alpha](\phi\to[\alpha]\phi)$
$[\alpha, \alpha](\phi\to[\alpha]\phi)$
$\vdots$

\section{Karo}
Opertunity vs capability: I have the opertunity to climb the dom tower\ldots

\subsection{KBD}
Knowledge Believe Desires.

$R_K = S5$
$R_B = KD45$
$R_B \subset \R_K = \vDash K\phi \to B\phi$
$R_D =$ no constraints, so sytem K.

infinite computation = $\nill$???

$R_\alpha(\nill)=\nill$ {divergence failure, or abort}

A = Agent is able to
$A\alpha \LeftRightarrow C(M,w)$

$<\phi?>\psi \LeftRightarrow (\psi \wedge \phi)$, if there is an accassible world
were $\psi$ holds and $\phi$ can be tested then in the world $\phi$ and $\psi$ hold.

Holds in general:

\[[\alpha]\phi \wedge <\alpha>\top\to <\alpha>\phi\]

If $\alpha$ is deterministic.

\[ <\alpha>\phi \to [\alpha]\phi \wedge <\alpha>\top\]

\section{KARO}
\subsection{Default}

\subsection{Goal}
implementable = a posibility to reach $\phi$

A goal is a achievement goal, something which is not true
has to become true.

Maintanence goals want to keep something true.

PossIntend = a very important one. An agent can
transform a possible intent into an commitment.

\subsection{Update agenda}
\[Ag(M',w') = Ag(M,w)\cup\alpha\]
for all $m',w'$ with $R(w',w'')$ 
\[Ag(M',w'') = Ag(M',w')\cup\alpha\]
\vdots

The agent is awere off what is going on. It knows
about the commitments.

you can only commit possible intents

You can only uncommit commited actions.
\end{document}
