#Bratman

Commitment, if you have a real intention, you want to go for it.
intentions should be consistent, otherwise you do something which could be
impossible
But this is debatable...


## Intentions
Screen of admissibility, filter maybe later intentions that could be
conflicitng


# Cohen& levesque
tried to formalise Bratman.

## Intention = choice + commitment
try to stretch it to formal logic.

##
agent beleives P is possible : a sort of self conficience of the agent

## side effects
agnets need not intend all the expected side effects of their intentions.
Going to the dentist implies pain.
then going to the dentist should be that you intend to have pain.

$\vDash \phi \to \psi \to \vDash \Box \phi \to \Box \psi$

#Rao & Georgeff BDI theory
Not only proposed a logic, but also the first implementation of an agent and
also the first application.

#Rodney Brooks anti-cogn.robotics
subsumption architecture: no cognition, just a reactive system.
Should be emergent behaviour rather than modeling.

# Aaron Sloman: How to build emotional agents
Looks at properties of humans, such as emotions, and tries to put it into an
AI.

20 years ago, there are patiens that don't have emotions, expect: in problem
solving such as puzzels. They go depth first, so they go on, and on and fail
misserably.

Archicterus of him are not the most enlighted, "messy".

# Towards social agents/robots
also research goes to sociology.

Oppertunisitic behavior, sellar buying a cup, but cup is broken,
the seller wants to sell it, but the buyer wants to use it. How
to moddel this in Agents

## Normative systems
the way of societies. you shouldn't violate the norm, but sometimes it can
be usefull to.

### Asimov
Laws of him. 4th one is stupid

# Theory of mind
The psycology one.
The false believe test.


# Logial foundations of actions

The terminology is not stabels.
desires and goals may be differetn or mean the same.

Agenda -> latin things that should happen.

# Modal logic
Logic of $\Box$ and $\Diamond$

## K axiom
true for any modal logic

## R
means its accesible. No restriction on R in arbitrary Kripke


### Constraints
you can add constraints on R, but with restrictions things certain
things become true.

serial you can go on.

## Frames
Models where the true assignment functions are abstracted from.

#naive BDI logic
seperate blocks, it misses connections.
No dymancis, changes of BDI in time.

# Cohen & levesque : langauge
$\alpha$ are complex actions and a's are primitive actions
We can't say multiple agents do the same action at the same time.

a is interpeted as a sequance of primitive events even though a is
atomic.

atomic action is a special class, which depends on the domain.

; = sequence
? = test if its true (groovy truth)
+ = is a non detrmenistic choice, collect them in an autcome set.
* = is done a finite but arbitrary number of times, including 0

if $\phi$ then $\alpha_1$ else
$\alpha_2$ = ($\phi? ; \alpha_1) + (\neg \phi?;\alpha)$

while $\phi$ do $\alpha$ = $(\phi?;\alpha)^*;\neg\phi?$

Pz = z has property P, a predicate attom


# Semantics
\omega = variables assginemnts
P = is the set of omega \in \omega
E = is a set of primitive events
Agt = Agt \in [E \to p] specifies the agent of an event [E \to P] is a
function that takes an E that gives a P.
T \subset [Z \to E]; a set of possible worlds, give a time number and get



