\abstract{
	%TODO abstract
}


%TODO Intro blurb, if any

\section{Introductions}

Satisficing play was originally introduced
to try and limit the computation power that earlier models required of human
behavior \citep{herbert}. Originating from the field of ecnomics, they
wanted to use a model that was not ``super intelligent''.

Such assumptions can include:

\begin{itemize}
  \item The structure of the game is known.
  \item Awareness of other player's decisions.
  \item Awareness of being in a game situation
  \item Being able to preform maximization calculations.
\end{itemize}

\section{Algorithm}
The core algorithm as presented by \citep{herbert} and formalized by
\citep{karandikar}
can be seen bellow:

\begin{enumerate}
  \item $ \begin{cases}
              \text{if } \pi_t \geq \alpha_t & A_{t+1} = A_t\\
              \text{else} & A_{t+1} \neq A_t
              \end{cases}$
  \item $ \alpha_{t+1} = \lambda \alpha_t + (1-\lambda) \pi_t $ \\
\end{enumerate}
\begin{tabular}{ll}
  Definition & symbol \\ \toprule
  Time & $t$ \\
  Action & $A_t$ \\
  Utility & $\pi_t$ \\
  Aspiration & $\alpha_t$ \\
  Learning rate & $\lambda$ \\ \bottomrule
\end{tabular}

In the first phase the next action is selected. If the action still holds
according to the satificing level or ambition it will remain the same, if not another
action will be selected. In the second phase the satisficing level is updated.
In here the constant learning rate is used. This is a variable that can be
modified to adjust the amount of new information being used or how much
information from memory is used.


\section{Iterated prisoners dilemma}

To demonstrate a potential use of this algorithm we will use at several example
cases. The first one being the iterated prisoners dilemma. Consider
the following normal form game:

\begin{tabular}{lll}
            & Cooperate   & Defect \\
  Cooperate & $(2,2)$     & $(0,3)$  \\
  Defect    & $(3,0)$     & $(1,1)$  \\
\end{tabular}

The Nash equilibrium of this games lies on Defect,Defect. However, all but
defect are pareto optimal actions.

\section{References to treat}
\begin{itemize}
  \item \citep{stimpson:2001}
  \item \citep{stimpson:2003}
  \item \citep{sandholm} % q-learning in iterated prisoners dilemma
  \item \citep{arthur} % maybe not, comparing algorithms against humans
  \item \citep{bowling}
  \item \citep{feltovich}
  \item \citep{crandall}
\end{itemize}

\bibliography{main}

\address{Jappie Klooster\\
  Dept. of Informatics\\
  Universiteit Utrecht\\
  The Netherlands\\}
\email{j.t.klooster@students.uu.nl}

