\abstract{
  %TODO abstract
  \todo[inline]{abstract missing}
}
\section{Introduction}
\toReview
Satisficing play was originally introduced
to try and limit the computation power that earlier models required of human
behavior \citep{herbert}. Originating from the field of economics, they
wanted to use a model that was not ``super intelligent''.

The classical models from the field of economics had at least one of the
following assumptions:
\begin{itemize}
\itemsep0em 
  \item The structure of the game is known.
  \item Awareness of other player's decisions.
  \item Awareness of being in a game situation
  \item Being able to preform maximization calculations.
\end{itemize}

These assumptions can be called common knowledge, and are a demanding to
make \citep{binmore}. 

\section{Core algorithm}
The core algorithm as presented by \citep{herbert} and formalized by
\citep{karandikar} \todo[inline]{karandikars algorithm is different..}
can be seen bellow:

\begin{enumerate}
  \item $ \begin{cases}
              \text{if } \pi_t \geq \alpha_t & A_{t+1} = A_t\\
              \text{else} & A_{t+1} \neq A_t
              \end{cases}$
  \item $ \alpha_{t+1} = \lambda \alpha_t + (1-\lambda) \pi_t $ \\
\end{enumerate}
\begin{tabular}{ll}
  Definition & symbol \\ \toprule
  Time & $t$ \\
  Action & $A_t$ \\
  Utility & $\pi_t$ \\
  Aspiration & $\alpha_t$ \\
  Learning rate & $\lambda$ \\ \bottomrule
\end{tabular}

In the first phase the next action is selected. If the action still holds
according to the satisficing level or ambition it will remain the same, if not another
action will be selected. In the second phase the satisficing level is updated.
In here the constant learning rate is used. This is a variable that can be
modified to adjust the amount of new information being used or how much
information from memory is used.


\section{Iterated prisoners dilemma}
\drafting
To demonstrate a potential use of this algorithm we will use at several example
cases. The first one being the iterated prisoners dilemma. Consider
the following normal form game:

\begin{tabular}{lll}
            & Cooperate   & Defect \\
  Cooperate & $(2,2)$     & $(0,3)$  \\
  Defect    & $(3,0)$     & $(1,1)$  \\
\end{tabular}

The Nash equilibrium of this games lies on Defect,Defect. However, all but
that action are pareto optimal actions. So the Nash equilibrium would be the
``right'' strategy to play if the game is only played once, but if you repeat
the game against the same opponent, interestingly enough you can ``solve''
the prisoners dilemma by ``teaching'' the opponent to cooperate.

However strategies such as tit for tat or win stay lose shift already know
which actions are winning and which are punishing to try and solve it.
Satisficing play learns this however by updating the aspiration it got from
playing a move. If we consider a game vs always cooperate for example:

\todo{copy table}

But now if we start with a lower apsiration level:

\todo{copy table}

And finally vs tit for tat, a game which can also be found in the
\citep{stimpson:2001} paper:

\todo{copy table}
\section{Properties of satisifing play}
\drafting

Using both \citep{karandikar} and \citep{stimpson:2001} works to look deeper
at the properties of satisifing play. 

karandikar used a theoretical proofing method, wehreas stimpson showed some
experimental results.

also showing the contrast graph which supports the work of \citep{arthur}

\section{An extended satisficing play}
\drafting

In here we discuss the work of \citep{stimpson:2003}, where he extends the
algorithm to be played with mutliple action choices.
Stimpson created a game where resources had to be allocated among agents.


\begin{tabular}{ll}
  Definition & symbol \\ \toprule
  Time & $t$ \\
  Action & $A_t$ \\
  Utility & $\pi_t$ \\
  Aspiration & $\alpha_t$ \\
  Learning rate & $\lambda$ \\ \bottomrule
\end{tabular}

\section{Satisficing compared to other learning methods}
\drafting
With help of the work of \citep{crandall} saticificing play is compared
to Q-learning and WOLF-PHC.
Q-learning is explained briefly with help of \citep{sandholm} and WOLF-PHC
with help of \citep{bowling}.
 
Finally some graphs are shown how they compare

\bibliography{main}

\address{Jappie Klooster\\
  Dept. of Informatics\\
  Universiteit Utrecht\\
  The Netherlands\\}
\email{j.t.klooster@students.uu.nl}
