\documentclass{article}
\usepackage{algorithmicx}
\usepackage{algpseudocode}
\usepackage{enumerate}
\renewcommand{\thesubsection}{\thesection.\alph{subsection}}
\algrenewcomment[1]{\(\triangleright\) #1}
\begin{document}
\author{Jappie Klooster}
\title{A summery of the interactive learning problem}

\maketitle

Equilibrium is very important to social systems. The nearest
thing we have as a fully developed framework for the study of
equilibria in social systems is Bayesian decision theory.

However there are difficulties in using this. Firstly it requires that
everyone has to anticipate correctly the behaviours of others.
Secondly it requires that individuals improve their strategies
at each point in time as events unfold.
Both these things are incredibility difficult if not impossible to do.

The reason why this only affects social systems and not physical systems
(which also desire equilibria) is that they're self referential.
The act of learning changes the thing to be learned.
% do double slit to confirm

%now lets talk about not knowing about copy on write
These difficulties posed by interactive learning effect many areas of economics.
There is an enormous informational burden on the adjustment
system of certain price convergence mechanism. This means
that truly decentralized convergent mechanisms don't grow
on trees.

These issues also plague macroeconomics. The question
is weather agents can learn the expected equilibrium from
out not in equilibrium data.
A proposed dumbed version system is a real valued variable,
for example next periods price level, is determined by current
and historical price level and also by people's expectation
about next period.

However convergence in such a system can be sensitive to the
nature of the learning process.

These issues also show up in game theory. Interaction
between expectation and behaviors is even more starkly present.
Some people think that games may be learned by simple updating
procedures, while others think this is not possible, because
this only works in special cases such as zero sum games.
They also think the Bayesian methods that lead to learning
are bad, because it requires to much computation and the
coordination required to do this requires some pre equilibrium
in the believe base. (The agents already assumed it was
possible and were willing to do their bestest best).

There will be Bayesian forced Nash equilibria, however there is
a limitation to this, because it assumes that players already
have partial knowledge of each others strategies.

Uncertain games create many issues. Its hard for example
to guess what the preference is based on just passed play
if you try to coordinate simultaneously.
\end{document}
