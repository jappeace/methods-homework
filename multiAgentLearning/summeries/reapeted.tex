\documentclass{article}
\usepackage{amsmath}
\begin{document}
\author{Jappie Klooster}
\title{A brief summery of repeated}

\maketitle

Within the prisoners dilemma game the Pareto inefficient outcome
cannot be avoided in a Nash equilibrium. This summery considers
two person infinitely repeated games, subgame perfect
equilibrium and Nash equilibrium.

\[G_p= 
	\bordermatrix{~ & C & D \cr
                  C & 50,50 & 30,60 \cr
                  D & 60,30 & 40,40 \cr}
\]

$G_p$ is a prisoners dilemma game. The strict dominant strategy
for each player is $D$. So $(D,D)$ is the Nash equilibrium of this game.
If $G_p$ is played infinitely many times $t=0,1,2\dots$ and each player
learns what element of the set $\{(C,C),(C,D), (D,C),(D,D)\}$ has occurred.
This induces an infinite stream of associated payoffs and it assumed that
there is a common discount factor $0 < \delta < 1$ so that the final payoff
to each player is $\delta$-discounted value of the infinite stream of
payoffs. Player i(=1,2) obtains:

\[\sum^\infty_{t=0}\delta^t_i\cdot(\text{payoff from $t$-th play})\]

Where $\delta_i=\delta$ for $i=1,2$. A strategy should prescribe an
action for a player at each moment in the history of the game up
to time $t$ of the set $\{(C,C),(C,D),(D,C),(D,D)\}$. The infinite
extensive form game just defined is denoted by $G^\infty_p(\delta)$.
A natural solution to this game is the concept of subgame perfect
equilibrium, where each subgame is equal to the game $G^\infty_p(\delta)$
itself.

A trigger strategy remains the same until some action occurs which
will change the strategy permanently. So you keep playing a cooperating
strategy until your opponent deviates from cooperating, to which you respond
with eternal punishment after that. With common sense we can determine that
such a strategy can solve the prisoners dilemma. This result is an important
reason to study infinitely repeating games.

By using more complex Trigger strategies utility can be increased even further
Such as for example an alternating trigger strategy, where you keep alternating
between playing $C$ and $D$, but as soon as your moves align with the opponent
you play $D$ forever.

If $G$ is an arbitrary $n-$person game, and the strategy combination
$s=(s_1,\dots,s_n)$ is a Nash equilibrium in $G$, If $0<\delta<1$ then each
player $i$ playing $s_i$ at every moment $t$ is a subgame perfect
equilibrium in $G^\infty(\delta)$.

The Folk theorem describes what happens with trigger strategies, players
keep cooperating with each other under the threat of being punished forever.

Even if you relax the possible game by not creating a Nash equilibrium, it
still pays of for players to cooperate with each other (under the threat
of playing the lesser profitable mixed strategy), since when doing
this they can increase their utility.
\paragraph{Questions}
\begin{itemize}
	\item What is a bimatrix game?
	\item Is the convex hull of $P(G)$ the same convex hull of that of a set of 
		points as I've encountered in geometric algorithms
		(ie the points that lie on the boundary of a shape)?
\end{itemize}

\end{document}
