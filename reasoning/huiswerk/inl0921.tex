\documentclass{article}
\usepackage{booktabs}
\usepackage{amsmath}
\usepackage{color}

\begin{document}
\author{J. T. Klooster}
\title{Homework of 21 september}
\maketitle

\section{Give proofs of the folowing two sequants in $GK$}
\subsection{$(A\to B\wedge C)\Rightarrow(A\to B)\wedge(A\to C)$}
\subsection{$\neg(A\wedge\neg A)$}
\section{Let P and Q be atoms. Give proofs in GK of the threee sequents}
\subsection{$(P\wedge Q \Rightarrow P \wedge Q)$}
\subsection{$(P\vee Q \Rightarrow P \vee Q)$}
\subsection{$(P \to Q \Rightarrow P \to Q$}
\section{Explain that $GK$ minus the Cut Rule has the \emph{subformula property}}
\subsection{Give an explanation that $GK$ (with the Cut Rule) does not have
the subformula property}
\section{Give the versions of $L \to, L\wedge,$ and Cut with split context}
\section{Let $GK_{split}$ be the sequent calcules $GK$ in which the rules}

\end{document}
