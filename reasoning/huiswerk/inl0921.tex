\documentclass{article}
\usepackage{booktabs}
\usepackage{amsmath}
\usepackage{color}
\usepackage{bussproofs}

\begin{document}
\author{J. T. Klooster}
\title{Homework of 21 september}
\maketitle

\section{Give proofs of the folowing two sequants in $GK$}
\subsection{$(A\to B\vee C)\Rightarrow(A\to B)\vee(A\to C)$}
Using GK and assuming A B and C are atoms the following tree came to be:

\begin{prooftree}
\AxiomC{$A \Rightarrow A$}
\RightLabel{$LW$}
\UnaryInfC{$A, A\Rightarrow A$}
\RightLabel{$RW$}
\UnaryInfC{$A, A\Rightarrow A$}
\RightLabel{$RW$}
\UnaryInfC{$A, A\Rightarrow A, C$}
\AxiomC{$B \Rightarrow B$}
\RightLabel{$LW$}
\UnaryInfC{$B, A\Rightarrow B$}
\RightLabel{$LW$}
\UnaryInfC{$B, A, A\Rightarrow B$}
\RightLabel{$RW$}
\UnaryInfC{$B, A, A\Rightarrow B, C$}
\AxiomC{$C \Rightarrow C$}
\RightLabel{$LW$}
\UnaryInfC{$C, A \Rightarrow C$}
\RightLabel{$LW$}
\UnaryInfC{$C, A, A\Rightarrow C$}
\RightLabel{$RW$}
\UnaryInfC{$C, A, A\Rightarrow B, C$}
\RightLabel{$L\vee$}
\BinaryInfC{$(B\vee C), A, A\Rightarrow B, C$}
\RightLabel{$L\to$}
\BinaryInfC{$(A\to B\vee C), A, A\Rightarrow B, C$}
\RightLabel{$R\to$}
\UnaryInfC{$(A\to B\vee C), A\Rightarrow B,A\to C$}
\RightLabel{$R\to$}
\UnaryInfC{$(A\to B\vee C)\Rightarrow A\to B,A\to C$}
\RightLabel{$R\vee$}
\UnaryInfC{$(A\to B\vee C)\Rightarrow(A\to B)\vee(A\to C)$}
\end{prooftree}

\subsection{$\neg(A\vee\neg A) \Rightarrow$}
Using GK and assuming A is an atom the following tree came to be:
\begin{prooftree}
\AxiomC{$A\Rightarrow A$}
\RightLabel{$R\neg$}
\UnaryInfC{$\Rightarrow A,\neg A$}
\RightLabel{$R\vee$}
\UnaryInfC{$\Rightarrow A\vee\neg A$}
\RightLabel{$L\neg$}
\UnaryInfC{$\neg(A\vee\neg A) \Rightarrow$}
\end{prooftree}

\section{Let P and Q be atoms. Give proofs in GK of the threee sequents}
\subsection{$(P\wedge Q \Rightarrow P \wedge Q)$}
\subsection{$(P\vee Q \Rightarrow P \vee Q)$}
\subsection{$(P \to Q \Rightarrow P \to Q$}
\section{Explain that $GK$ minus the Cut Rule has the \emph{subformula property}}
\subsection{Give an explanation that $GK$ (with the Cut Rule) does not have
the subformula property}
\section{Give the versions of $L \to, L\wedge,$ and Cut with split context}
\section{Prove that $GK_{split}$ and $GK$ prove the same sequants}
\section{Prove that $GK$ is sound: If $\vdash_GK S$, then $I(S)$ is true.}

\end{document}
