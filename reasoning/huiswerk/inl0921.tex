\documentclass{article}
\usepackage{booktabs}
\usepackage{amsmath}
\usepackage{color}
\usepackage{bussproofs}

\begin{document}
\author{J. T. Klooster}
\title{Homework of 21 september}
\maketitle

\section{Give proofs of the folowing two sequants in $GK$}
\subsection{$(A\to B\vee C)\Rightarrow(A\to B)\vee(A\to C)$}
Using GK and assuming A B and C are atoms the following tree came to be:

\begin{prooftree}
\AxiomC{$A \Rightarrow A$}
\RightLabel{$LW$}
\UnaryInfC{$A, A\Rightarrow A$}
\RightLabel{$RW$}
\UnaryInfC{$A, A\Rightarrow A$}
\RightLabel{$RW$}
\UnaryInfC{$A, A\Rightarrow A, C$}
\AxiomC{$B \Rightarrow B$}
\RightLabel{$LW$}
\UnaryInfC{$B, A\Rightarrow B$}
\RightLabel{$LW$}
\UnaryInfC{$B, A, A\Rightarrow B$}
\RightLabel{$RW$}
\UnaryInfC{$B, A, A\Rightarrow B, C$}
\AxiomC{$C \Rightarrow C$}
\RightLabel{$LW$}
\UnaryInfC{$C, A \Rightarrow C$}
\RightLabel{$LW$}
\UnaryInfC{$C, A, A\Rightarrow C$}
\RightLabel{$RW$}
\UnaryInfC{$C, A, A\Rightarrow B, C$}
\RightLabel{$L\vee$}
\BinaryInfC{$(B\vee C), A, A\Rightarrow B, C$}
\RightLabel{$L\to$}
\BinaryInfC{$(A\to B\vee C), A, A\Rightarrow B, C$}
\RightLabel{$R\to$}
\UnaryInfC{$(A\to B\vee C), A\Rightarrow B,A\to C$}
\RightLabel{$R\to$}
\UnaryInfC{$(A\to B\vee C)\Rightarrow A\to B,A\to C$}
\RightLabel{$R\vee$}
\UnaryInfC{$(A\to B\vee C)\Rightarrow(A\to B)\vee(A\to C)$}
\end{prooftree}

\subsection{$\neg(A\vee\neg A) \Rightarrow$}
Using GK and assuming A is an atom the following tree came to be:
\begin{prooftree}
\AxiomC{$A\Rightarrow A$}
\RightLabel{$R\neg$}
\UnaryInfC{$\Rightarrow A,\neg A$}
\RightLabel{$R\vee$}
\UnaryInfC{$\Rightarrow A\vee\neg A$}
\RightLabel{$L\neg$}
\UnaryInfC{$\neg(A\vee\neg A) \Rightarrow$}
\end{prooftree}

\section{Let P and Q be atoms. Give proofs in GK of the threee sequents}
\subsection{$(P\wedge Q \Rightarrow P \wedge Q)$}
\begin{prooftree}
\AxiomC{$P\Rightarrow P$}
\RightLabel{$RW$}
\UnaryInfC{$P, Q \Rightarrow P$}
\AxiomC{$Q\Rightarrow Q$}
\RightLabel{$RW$}
\UnaryInfC{$P, Q \Rightarrow Q$}
\RightLabel{$R\wedge$}
\BinaryInfC{$P, Q \Rightarrow P \wedge Q$}
\RightLabel{$L\wedge$}
\UnaryInfC{$P\wedge Q \Rightarrow P \wedge Q$}
\end{prooftree}
\subsection{$(P\vee Q \Rightarrow P \vee Q)$}
\begin{prooftree}
\AxiomC{$P\Rightarrow P$}
\RightLabel{$LW$}
\UnaryInfC{$P\Rightarrow P, Q$}
\AxiomC{$Q\Rightarrow Q$}
\RightLabel{$LW$}
\UnaryInfC{$Q \Rightarrow P, Q$}
\RightLabel{$L\vee$}
\BinaryInfC{$P\vee Q \Rightarrow P, Q$}
\RightLabel{$R\vee$}
\UnaryInfC{$P\vee Q \Rightarrow P \vee Q$}
\end{prooftree}
\subsection{$(P \to Q \Rightarrow P \to Q)$}
\begin{prooftree}
\AxiomC{$P \Rightarrow P$}
\RightLabel{$RW$}
\UnaryInfC{$P \Rightarrow P, Q$}
\AxiomC{$Q\Rightarrow Q$}
\RightLabel{$LW$}
\UnaryInfC{$P, Q\Rightarrow Q$}
\RightLabel{$L\to$}
\BinaryInfC{$(P\to Q), P \Rightarrow Q$}
\RightLabel{$R\to$}
\UnaryInfC{$P \to Q \Rightarrow P \to Q$}
\end{prooftree}
\section{Explain that $GK$ minus the Cut Rule has the \emph{subformula property}}
The subformula porperty means that you can't change a formula beyond the root
formula.
If you use GK without the Cut Rule, it is impossible to transpose a formula into
a different one or the atom count stays the same or becomes less.
\subsection{Give an explanation that $GK$ (with the Cut Rule) does not have
the subformula property}
Because the Cut rule allows you to add atoms. Thus the formula can become
very different from the root formula.
\newpage
\section{Give the versions of $L \to, L\vee$ and Cut with split context}
\begin{prooftree}
\AxiomC{$\Gamma_1, \Gamma_2 \Rightarrow A, \Delta_1,\Delta_2$}
\AxiomC{$\Gamma_1, \Gamma_2, B \Rightarrow \Delta_1,\Delta_2$}
\RightLabel{$L\to$}
\BinaryInfC{$\Gamma_1, \Gamma_2, A \to B \Rightarrow \Delta_1,\Delta_2$}
\end{prooftree}
\begin{prooftree}
\AxiomC{$\Gamma_1, \Gamma_2, A \Rightarrow \Delta_1,\Delta_2$}
\AxiomC{$\Gamma_1, \Gamma_2, B \Rightarrow \Delta_1,\Delta_2$}
\RightLabel{$L\vee$}
\BinaryInfC{$\Gamma_1, \Gamma_2, A \vee B \Rightarrow \Delta_1,\Delta_2$}
\end{prooftree}
\begin{prooftree}
\AxiomC{$\Gamma_1, \Gamma_2 \Rightarrow A, \Delta_1,\Delta_2$}
\AxiomC{$\Gamma_1, \Gamma_2, A \Rightarrow \Delta_1,\Delta_2$}
\RightLabel{Cut}
\BinaryInfC{$\Gamma_1, \Gamma_2, \Rightarrow \Delta_1,\Delta_2$}
\end{prooftree}
\section{Prove that $GK_{split}$ and $GK$ prove the same sequants}
The only things that has been moddified on $GK_{split}$ compared to $GK$ are
$R\wedge, L \to, L\vee$ and the Cut rule.
If the precedents can be transformed from the one system to the other
system and the other way around than the systems can proof the same
sequants.

\subsection{$R\wedge$}
\subsubsection{From $GK$ to $GK_{split}$}
\begin{prooftree}
\AxiomC{$\Gamma_1, \Gamma_2 \Rightarrow A \wedge B,  \Delta_1,\Delta_2$}
\UnaryInfC{$\Gamma  \Rightarrow A \wedge B, \Delta$}
\end{prooftree}
\subsubsection{From $GK_{split}$ to $GK$}
\begin{prooftree}
\AxiomC{$\Gamma  \Rightarrow A \wedge B, \Delta$}
\UnaryInfC{$\Gamma_1, \Gamma_2 \Rightarrow A \wedge B,  \Delta_1,\Delta_2$}
\end{prooftree}
\subsection{$L\to$}
\subsubsection{From $GK$ to $GK_{split}$}
\begin{prooftree}
\AxiomC{$\Gamma_1, \Gamma_2, A \to B \Rightarrow \Delta_1,\Delta_2$}
\UnaryInfC{$\Gamma, A \to B \Rightarrow \Delta$}
\end{prooftree}
\subsubsection{From $GK_{split}$ to $GK$}
\begin{prooftree}
\AxiomC{$\Gamma, A \to B \Rightarrow \Delta$}
\UnaryInfC{$\Gamma_1, \Gamma_2, A \to B \Rightarrow \Delta_1,\Delta_2$}
\end{prooftree}
\subsection{$L\vee$}
\subsubsection{From $GK$ to $GK_{split}$}
\begin{prooftree}
\AxiomC{$\Gamma_1, \Gamma_2, A \vee B \Rightarrow \Delta_1,\Delta_2$}
\RightLabel{LC}
\UnaryInfC{$\Gamma,A \vee B \Rightarrow \Delta_1,\Delta_2$}
\RightLabel{RC}
\UnaryInfC{$\Gamma, A \vee B \Rightarrow \Delta$}
\end{prooftree}
\subsubsection{From $GK_{split}$ to $GK$}
\begin{prooftree}
\AxiomC{$\Gamma,A \vee B \Rightarrow \Delta$}
\RightLabel{RW}
\UnaryInfC{$\Gamma,A \vee B\Rightarrow \Delta_1,\Delta_2$}
\RightLabel{LW}
\UnaryInfC{$\Gamma_1, \Gamma_2, A \vee B \Rightarrow \Delta_1,\Delta_2$}
\end{prooftree}
\subsection{Cut}
\subsubsection{From $GK$ to $GK_{split}$}
\begin{prooftree}
\AxiomC{$\Gamma_1, \Gamma_2 \Rightarrow \Delta_1,\Delta_2$}
\RightLabel{LC}
\UnaryInfC{$\Gamma \Rightarrow \Delta_1,\Delta_2$}
\RightLabel{RC}
\UnaryInfC{$\Gamma \Rightarrow \Delta$}
\end{prooftree}
\subsubsection{From $GK_{split}$ to $GK$}
\begin{prooftree}
\AxiomC{$\Gamma \Rightarrow \Delta$}
\RightLabel{RW}
\UnaryInfC{$\Gamma \Rightarrow \Delta_1,\Delta_2$}
\RightLabel{LW}
\UnaryInfC{$\Gamma_1, \Gamma_2  \Rightarrow \Delta_1,\Delta_2$}
\end{prooftree}
\section{Prove that $GK$ is sound: If $\vdash_GK S$, then $I(S)$ is true.}

\end{document}
