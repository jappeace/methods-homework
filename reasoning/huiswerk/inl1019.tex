\documentclass{article}
\usepackage{txfonts}
\usepackage{booktabs}
\usepackage{color}
\usepackage{bussproofs}
\usepackage{graphicx}
\usepackage{pifont}
\newenvironment{scprooftree}[1]%
{\gdef\scalefactor{#1}\begin{center}\proofSkipAmount \leavevmode}%
{\scalebox{\scalefactor}{\DisplayProof}\proofSkipAmount \end{center} }

\begin{document}
\author{Jappie Klooster}
\title{On proving things}
\maketitle

\section{Proof GKC rules}
\subsection{$L\to$}
To proof that if one of the premises of $L\to$ of $GKC$ is not true
(under all valuations) than the conclusion not true either.

The definition of $L\to$ in $GKC$:

\begin{scprooftree}{1}
	\AxiomC{$\Gamma \Rightarrow A, \Delta$}
	\AxiomC{$\Gamma, B \Rightarrow \Delta$}
	\RightLabel{$L\to$}
	\BinaryInfC{$\Gamma, A \to B \Rightarrow \Delta$}
\end{scprooftree}

Let $V(x)$ be the valuation function that determines if sequent x
is true.
Let $\Gamma \Rightarrow A, \Delta$ be $P_1$,
let $\Gamma, B \Rightarrow \Delta$ be $P_2$ and
let $\Gamma, A \to B \Rightarrow \Delta$ be $C$

To proof $V(P_1) = 0$ or
$V(P_2) = 0$ then
$V(C) = 0$. We will try to proof this by searching for a case
where one of the $P$'s is false and $C$ is true.

Both $V(P_1)$ and $V(P_2)$ can only be 0 if the result of
$V(\Gamma) \not= V(\Delta)$
so we assume $V(\Gamma) = 1$ and $V(\Delta) = 0$.

If $V(A) = 1$ in this case $V(P_1)$ becomes true so $V(P_2)$ should be false.
The only way to make $P_2$ false is by having $V(B) = 1$.
But once you do this $V(A \to B) = 1$ which means $V(C) = 0$

The other option is to make $V(A) = 0$ Now $V(P_1)$ is false, so $V(P_2)$ can
be true or false. The value of $B$ doesn't matter because in either case
$V(A \to B) = 1$ which means $V(C)$ will be false.

The same technique can be applied for $V(\Gamma) = 0$ and $V(\Delta) = 1$.

There are no other valuations left, so if one of the premises of L $\to$ is
false then so is the conclusion

\subsection{$R\to$}
To proof that if one of the premises of $R\to$ of $GKC$ is not true
(under all valuations) than the conclusion not true either.

The definition of $R\to$ in $GKC$:

\begin{scprooftree}{1}
	\AxiomC{$\Gamma, A \Rightarrow B, \Delta$}
	\RightLabel{$R\to$}
	\UnaryInfC{$\Gamma \Rightarrow A \to B, \Delta$}
\end{scprooftree}
Let $V(x)$ be the valuation function that determines if sequent x
is true.
Let $\Gamma, A \Rightarrow B, \Delta$ be $P$ and
let $\Gamma \Rightarrow A \to B, \Delta$ be $C$


$V(P)$ can only be 0 if the result of
$V(\Gamma) \not= V(\Delta)$

If $V(\Gamma) = 1$, $V(A)$ also has to be 1 otherwise the premise would be true.
$V(B)$ has to be 0 because otherwise the premise would be true.

With these valuation $V(C)$ will be false.

The same result for $V(\Gamma) = 0$

There are no other valuations left, so if one of the premises of R $\to$ is
false then so is the conclusion


\section{Positivity and negativity}
\subsection{Prove C keeps occuring in premise and conclusion}
To prove that if formula $C$ occurs positively in one of the premises of a rule,
then the same formula occurs positively in the conclusion for all rules of GKC

This will be done on a rule by rule basis.

\subsubsection{$R\wedge$}
Suppose C occurs positively in one of the premisses
($\Gamma \Rightarrow A, \Delta$) or ($\Gamma \Rightarrow B,\Delta$),
say in the first one. Thus it occurs positively in a formula in $\Delta$ or
$A$ or negatively in a formula in $\Gamma$. In the first and the last case,
the same holds for the conclusion ($\Gamma \Rightarrow A \wedge B, \Delta$).
In case C occurs positively in $A$ then by definition it occurs positively
in $A \wedge B$ (even in the case that $C = A$). Hence it occurs positively
in ($\Gamma \Rightarrow A \wedge B, \Delta$). This is what had to be shown.

\subsubsection{$L\vee$}
Suppose C occurs positively in one of the premisses
($\Gamma, A \Rightarrow \Delta$) or ($\Gamma, B \Rightarrow \Delta$),
say in the first one. Thus it occurs positively in a formula in $\Delta$ or 
negatively in a formula in $\Gamma$ or $A$. In the first and the last case,
the same holds for the conclusion ($\Gamma, A \vee B \Rightarrow \Delta$).
In case C occurs negatively in $A$ then by definition it occurs negatively
in $A \vee B$ (even in the case that $C = A$). Hence it occurs negatively
in ($\Gamma, A \vee B \Rightarrow \Delta$). This is what had to be shown.

\subsubsection{$L\to$}
Suppose C occurs positively in one of the premisses
($\Gamma \Rightarrow A, \Delta$) or ($\Gamma, B \Rightarrow \Delta$),
say in the first one. Thus it occurs positively in a formula in $\Delta$ or
$A$ or negatively in a formula in $\Gamma$. In the first and the last case,
the same holds for the conclusion ($\Gamma A \to B \Rightarrow \Delta$).
In case C occurs positively in $A$ then by definition it occurs positively
in $A \to B$ (even in the case that $C = A$). Hence it occurs positively
in ($\Gamma A \to B \Rightarrow \Delta$). This is what had to be shown.

\subsubsection{$L\wedge$}
Suppose C occurs positively the premiss
($\Gamma, A, B \Rightarrow \Delta$). Thus it occurs positively in a formula
in $\Delta$ or negatively in a formula in $\Gamma$, $A$ or $B$.`
In the first and the last case,
the same holds for the conclusion ($\Gamma, A \wedge B \Rightarrow \Delta$).
In case C occurs negatively in $A$ then by definition it occurs negatively
in $A \wedge B$ (even in the case that $C = A$). Hence it occurs negatively
in ($\Gamma, A \wedge B \Rightarrow \Delta$). This is what had to be shown.

\subsubsection{$R\vee$}
Suppose C occurs positively the premiss
($\Gamma \Rightarrow A, B, \Delta$). Thus it occurs positively in a formula
in $\Delta$, $A$ or $B$ or negatively in a formula in $\Gamma$.
In the first and the last case,
the same holds for the conclusion ($\Gamma \Rightarrow A \vee B, \Delta$).
In case C occurs positively in $A$ then by definition it occurs positively
in $A \vee B$ (even in the case that $C = A$). Hence it occurs positively
in ($\Gamma, A \wedge B \Rightarrow \Delta$). This is what had to be shown.

\subsubsection{$R\to$}
Suppose C occurs positively the premiss
($\Gamma, A \Rightarrow B, \Delta$). Thus it occurs positively in a formula
in $\Delta$ or $B$ or negatively in a formula in $\Gamma$ or $A$.
In the first and the last case,
the same holds for the conclusion ($\Gamma \Rightarrow A \to B, \Delta$).
In case C occurs negatively in $A$ then by definition it occurs negatively
in $A \to B$ (even in the case that $C = A$). Hence it occurs negatively
in ($\Gamma, A \to B \Rightarrow \Delta$). This is what had to be shown.

\subsubsection{$L\neg$}
Suppose C occurs positively the premiss
($\Gamma \Rightarrow A, \Delta$). Thus it occurs positively in a formula
in $\Delta$ or $A$ or negatively in a formula in $\Gamma$.
In the first and the last case,
the same holds for the conclusion ($\Gamma, \neg A \Rightarrow \Delta$).
In case C occurs positively in $A$ then by definition it occurs positively
in $\neg A$ (even in the case that $C = A$). Hence it occurs positively
in ($\Gamma, \neg A \Rightarrow \Delta$). This is what had to be shown.

\subsubsection{$R\neg$}
Suppose C occurs positively the premiss
($\Gamma, A \Rightarrow \Delta$). Thus it occurs positively in a formula
in $\Delta$ or negatively in a formula in $\Gamma$ or $A$.
In the first and the last case,
the same holds for the conclusion ($\Gamma \Rightarrow \neg A, \Delta$).
In case C occurs negatively in $A$ then by definition it occurs negatively
in $\neg A$ (even in the case that $C = A$). Hence it occurs negatively
in ($\Gamma \Rightarrow \neg A, \Delta$). This is what had to be shown.

\subsection{Proof introduction remains positive}

\subsubsection{$A\wedge B$}
To prove that in any proof in GKC of a sequent $S$, if the formula
$A \wedge B$ is introduced in an application of the $R\wedge$ rule,
then that $A \wedge B$ occurs positively in the endsequent $S$of the
proof.

Let $A \wedge B$ be C, as shown in 2.1 if formula C occurs positively in
one of the premises of a rule, then the same formula occurs positively in
the conclusion for all rules of GKC, now we have proven that C will
occur in the endsequent.

\subsubsection{$A\vee B$}
To prove that in any proof in GKC of a sequent $S$, if the formula
$A \vee B$ is introduced in an application of the $R\vee$ rule,
then that $A \vee B$ occurs positively in the endsequent $S$of the
proof.

Let $A \vee B$ be C, as shown in 2.1 if formula C occurs positively in
one of the premises of a rule, then the same formula occurs positively in
the conclusion for all rules of GKC, now we have proven that C will
occur in the endsequent.

\subsection{Single branch}
To prove that every provable sequent that does not contains negations and
implications and in which every subformula of the form $A \wedge B$ occurs
negatively and every subformula of the form $A \vee B$ occurs positively,
has a proof in GKC that consists of a single branch.

Because the sequent does not contain negations and implications the rules
$R\to$, $L\to$, $R\neg$ and $L\neg$ can't be applied.
$R\wedge$ can't be applied because $A \wedge B$ occurs negatively,
which by definition means $A \wedge B$ occurs on the $L$ side of the
$\wedge$ rules.
$L\vee$ can't be
applied because $A \vee B$ occurs positively, which by definition means
that the $A \vee B$ formula is on the $R$ side of the $\vee$ rules.

The only applicable rules left in GKC are the $L\wedge$ and $R\vee$ rules.
These are single branch by definition. Thus this entire proof has a single
branch, this is what had to be shown.

\subsection{$S$ consists of \ldots}
Supose that $S$ is a provable sequent that does not contain negations and
implications and which every subformula of the form $A\wedge B$ occurs
negatively and every subformula of the form $A \vee B$ occurs positively.
Show that if the sum of all the occurrences of connectives in $S$ is $n$.
then $S$ has a proof in $GKC$ that consists of at most $n + 1$ sequents.


Because the sequent $S$ does not contain negations and implications the rules
$R\to$, $L\to$, $R\neg$ and $L\neg$ can't be applied.
$R\wedge$ can't be applied because $A \wedge B$ occurs negatively,
which by definition means $A \wedge B$ occurs on the $L$ side of the
$\wedge$ rules.
$L\vee$ can't be
applied because $A \vee B$ occurs positively, which by definition means
that the $A \vee B$ formula is on the $R$ side of the $\vee$ rules.

The only applicable rules left in GKC are the $L\wedge$ and $R\vee$ rules.
Both the $L\wedge$ and $R\vee$ rules add only 1 connective to the conclusion
once applied to the premise, creating a new sequent in the process. The proof
of $S$ also has just a single branch, because $L\wedge$ and $R\vee$ only allow
a single premise.
So the
sequent count in the proof of $S$ is at least $n$. It is $n$ + 1 because $S$ is also
a sequent by definition. This is what had to be shown.
\end{document}
