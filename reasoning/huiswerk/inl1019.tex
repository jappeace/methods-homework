\documentclass{article}
\usepackage{txfonts}
\usepackage{booktabs}
\usepackage{color}
\usepackage{bussproofs}
\usepackage{graphicx}
\usepackage{pifont}
\newenvironment{scprooftree}[1]%
{\gdef\scalefactor{#1}\begin{center}\proofSkipAmount \leavevmode}%
{\scalebox{\scalefactor}{\DisplayProof}\proofSkipAmount \end{center} }

\begin{document}
\author{Jappie Klooster}
\title{On proving things}
\maketitle

\section{Proof GKC rules}
\subsection{$L\to$}
To proof that if one of the premises of $L\to$ of $GKC$ is not true
(under all valuations) than the conclusion not true either.

The definition of $L\to$ in $GKC$:

\begin{scprooftree}{1}
	\AxiomC{$\Gamma \Rightarrow A, \Delta$}
	\AxiomC{$\Gamma, B \Rightarrow \Delta$}
	\BinaryInfC{$\Gamma, A \to B \Rightarrow \Delta$}
\end{scprooftree}

Let $V(x)$ be the valuation function that determines if sequent x
is true.

Let $\Gamma \Rightarrow A, \Delta$ be $P_1$,
let $\Gamma, B \Rightarrow \Delta$ be $P_2$ and
let $\Gamma, A \to B \Rightarrow \Delta$ be $C$

To proof $V(P_1) = 0$ or
$V(P_2) = 0$ then
$V(C) = 0$. We will try to proof this by searching for a case
where one of the $P$'s is false and $C$ is true.

Both $V(P_1)$ and $V(P_2)$ can only be 0 if the result of
$V(\Gamma) \not= V(\Delta)$
so we assume $V(\Gamma) = 1$ and $V(\Delta) = 0$.

If $V(A) = 1$ in this case $V(P_1)$ becomes true so $V(P_2)$ should be false.
The only way to make $P_2$ false is by having $V(B) = 1$.
But once you do this $V(A \to B) = 1$ which means $V(C) = 0$

The other option is to make $V(A) = 0$ Now $V(P_1)$ is false, so $V(P_2)$ can
be true or false. The value of $B$ doesn't matter because in either case
$V(A \to B) = 1$ which means $V(C)$ will be false.

The same technique can be applied for so we assume $V(\Gamma) = 0$ and $V(\Delta) = 1$.

\subsection{$R\to$}
\end{document}
