\documentclass{article}
\usepackage{txfonts}
\usepackage{booktabs}
\usepackage{color}
\usepackage{bussproofs}
\usepackage{graphicx}
\usepackage{pifont}
\usepackage{qtree}
\usepackage{tikz}
\usepackage{listings}
\usepackage{hyperref}
\newenvironment{scprooftree}[1]%
{\gdef\scalefactor{#1}\begin{center}\proofSkipAmount \leavevmode}%
{\scalebox{\scalefactor}{\DisplayProof}\proofSkipAmount \end{center} }

\begin{document}
\lstset{language=Java}
\author{Jappie Klooster}
\title{A Summery about reasoning with logic}
\maketitle

\section{Introduction}
This summery is written for people that try to understand logicians on a master
level *but* they're comming from `hbo informatica'

\section{On proofing things}
If you were lucky like me you'll be asked to write a proof in the first week
which will be graded. I got a terible grade, but learning I did.

So there are 2 kinds of proving. The first one is writing a proof in the system
this should be rather easy, although you have to be carefull to follow the
rules to the letter. I'm going to ignore this one since its just using the
rules, which you've done tons of times as a programmer, so it shouldn't be
that strange.

The second one is the more troublsome, its proving the system itself. This
is also split up into 3 different aspects. They all have in commen that
it should be done in natural langauge. But, there is a structure to it
I'm going to try and give a complete overview of this structure.

First you have soundness, then there is completness, and finally you can be
asked to proof certain properties of a system.

\subsection{functions}
Proving things is sometimes done with a function,
now you should be familiar with functions.
But they are defined in natural langauge, which may seem a little starnge
so as an example I'll define the valuation as first time I came accross it:

\[v:\mathcal{F}_{prp} \to \{0,1\}\]

the prototype in Java is like this (I assume you never can be sure of course,
see \autoref{notations}):


\begin{lstlisting}[frame=single] 
	public boolean V(Formula f);
\end{lstlisting}

Now in logic you describe the datatypes in natural langauge, so first they will
say `let', `define' or ` v is a function' \ldots

The other function I came across was the I function. Which stands for 
interpation, what it does is just replacing the commas for the propper
connectives.

\section{proving soundness}
This basically means your system doesn't modify the value you put in.
If you put in truth, apply a bunch of rules from your system, then you should
get out truth.

\subsection{Structure of your proof}
First write down your overal strategy of the proof \autoref{strategies}. For
soundness this is often `proof by induction' \autoref{induction}.

Then write down the lemmas you're gonna use. A lemma is a little proof, often
derived from the properties of the system you're proving. You could see it
as a basic argument. When proving soundness you want to connect this to the
semantics, for classical systems the semantics are truth tables, for kripke
models it are kripke frames. The semantics are the thing your talking about
in your logical system.
%I can see how I dind't understand all this crap the first time I saw it,
% its freaking packed with definitions, and even if you understand all the
% definiontions, its still hard
Make sure that you've proven the soundness of the axioms *and* of the rules
with help of the semantics.

Finally ty up the lemmas together in the conclusion, this is done with help
of the proofing strategy.

Finish with a sentence like: `This had to be shown'.

\section{proving strategies}
\label{strategies}
There are many different ways of writing proofs, in here I'll sumerize the ones
I came accross during the course.

\subsection{Induction}
\label{induction}


\section{notations}
\label{notations}
There is no standart in notating things. Every logician seems to use a
slightly different notations. Keep this in mind. But I think the connectives
are mostly the same. You probably need to memorize these.

\section{Systems}
GKC, Gentzen sequent calculy.

\end{document}
