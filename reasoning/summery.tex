\documentclass{article}
\usepackage{txfonts}
\usepackage{booktabs}
\usepackage{color}
\usepackage{bussproofs}
\usepackage{graphicx}
\usepackage{pifont}
\usepackage{qtree}
\usepackage{tikz}
\usepackage{listings}
\usepackage{hyperref}
\newenvironment{scprooftree}[1]%
{\gdef\scalefactor{#1}\begin{center}\proofSkipAmount \leavevmode}%
{\scalebox{\scalefactor}{\DisplayProof}\proofSkipAmount \end{center} }


\newcommand{\brcell}[2][l]{%
	\begin{tabular}[#1]{@{}l@{}}#2\end{tabular}}

\begin{document}
\lstset{language=Java}
\author{Jappie Klooster}
\title{A Summery about reasoning with logic}
\maketitle

\section{Introduction}
This summery is written for people that try to understand logicians on a master
level *but* they're comming from `hbo informatica'

\section{On proofing things}
If you were lucky like me you'll be asked to write a proof in the first week
which will be graded. I got a terible grade, but learning I did.

Proving is in meta space. Its a little like programming in that you have to
be specific. There are lots of ways to prove things. In this report I will
discuss a select few to give an idea about what is going on. But first
I'll start covering with some basics.

\subsection{Jargon}
In here I will discuss some jargon logicians use. It may seem silly but knowing
the precise defintion of these words makes understanding proofs a lot easier.

\begin{tabular}{ll}
	Jargon & Meaning\\ \toprule
	Formula & The letters, but also letters combined with conectives (its
	recursive) \\
	Axiom & \brcell{Something which is undoubtibly valid. Proofs work from a \\
	complex formula towards axioms.} \\
	Theorom & the thing you're trying to proof. \\
	Lemma & \brcell{a hypothosis which you're trying to prove, you usually
		define \\
	a lemma to reference to later. Its like a little theorom} \\
	Sequent & A thing with a sequent connective (in my case either
	$\vdash$ or $\Rightarrow$) \\
	Derivable & if X is derivable in system A, then you can proof x with
	system A
\end{tabular}


\subsection{functions}
Proving things is sometimes done with a function,
now you should be familiar with functions.
But they are defined in natural langauge, which may seem a little starnge
so as an example I'll define the valuation as first time I came accross it:

\[v:\mathcal{F}_{prp} \to \{0,1\}\]

the prototype in Java is like this (I assume you never can be sure of course,
see \autoref{notations}):


\begin{lstlisting}[frame=single] 
	public boolean V(Formula f);
\end{lstlisting}

Now in logic you describe the datatypes in natural langauge, so first they will
say `let', `define' or ` v is a function' \ldots

The other function I came across was the I function. Which stands for 
interpation, what it does is just replacing the commas for the propper
connectives.

\section{Structure of proofs}
\label{structure of proofs}
Now of course different poeple have different kinds of notations. The logicians
don't really seem to adhere to standarts. But that's fine, you just have to
get used to conflicting symbols.

\subsection{Find references}
Before you do anything check if you can't refer to any existing proofs.
This saves a lot of time. You can just ignore writing a proof if you can
write down a reference. (I bet this is allowed on the exam as long you
can refer to the stuff you wrote down yourself).

\subsection{Head of a proof}
In the head of the proof you don't do any work, you just state what we're
talking about and what you're going to do. So you start with the definitions.
Remeber the definitions should be stupid, only add obvious things, don't do
deductions until the body of the proof.
Then you say what you're going to proof (by using the defintions), so here
you just repeat the question. And finally you mention your strategy.

\subsection{Body of the proof}
In here you work out the proof with help of the strategy.
See the \autoref{strategies} section.

\subsection{End of the proof}
Just write down ` QED' or ` this is what had to be shown' even if you're not
certain what you're doing.

\section{proving strategies}
\label{strategies}
There are many different ways of writing proofs, in here I'll sumerize the ones
I came accross during the course.

\subsection{Induction}
\label{induction}
\subsubsection{When to use it?}
You can use induction if you have to proof something for countibly infinite
things, but there exists a base method that can validate if its true.
So basically all thigns that can and make sense be resolved with a
recursive function.

\subsubsection{How to use it?}
In here you try to prove something by starting with a single case n.
For all n (*), n=1, supose (*) for n, then show (*) for n+1.

Fat chance you'll understand it based on that. Its way to abstract.
But I'll show how to use it with help of several examples. \emph{cursive
text will be comments}

\paragraph{Proof that propositional formulas always have an even number of brackets.}
\emph{So space cowboy, we have to write a proof, before panicking remember
	the boiler plate in \autoref{structure of proofs}, we may get some
points for doing this, its at least polite}

\subparagraph{Definitions}
A propositional formula can be a propositional variable,
in which case it has zero brackets.
A propositional formula can also be multpile proporsitional variables,
connected with connectives, in this case it has the connective count * 2
brackets, because for every connective 2 brackets are added.

\subparagraph{Strategy}
For all propositional formula the number of brackets is even.
This will be proven with help of iniduction.

\subparagraph{Proof}
\emph{So this is where the induction starts, first do the base case}
The base case where n = 0,
a propositional formula is a single propositional variable. This means
it has 0 brackets according to definition 1, and 0 is an even number, so
the amount of brackets in the base case is even.

\emph{Induction step}
For the case n+1, asume $A$ and $B$ have a even number of brackets,
then $A$ connective $B$ has an even number of brackets to because you have
to add an even number of brackets according to the definition of adding
connectives.

\emph{Because this can be used for the case of propositional variables,
and all complexer formulas that can be constructed with connectives, you're
done. All posibilities have been treated now.}

QED


\paragraph{Proof that all natural numbers are either odd or even}
\emph{Another quite obvious prove, but this one has a more complex induction
step, so I'll treat it here}

\subparagraph{Definitions}
Natural numbers are all integers bigger than 0.

\[
Even(n) := \left\{
	\begin{array}{l}
		n= 0 \\
		Even(n-2)
	\end{array}
\right.
\]
\[
Odd(n) := \left\{
	\begin{array}{l}
		n= 1 \\
		Odd(n-2)
	\end{array}
\right.
\]

\subparagraph{Strategy}
All Natural numbers are either odd or even, this will be proven by induction.

\subparagraph{Proof} \emph{In here we look at the definitions and try to see
the base case(s)}. In case n=0, n is even according to the definition of
even, so n is either postivie or negative. \emph{reflecting back on what
we're trying to prove, saying its the case}

In case n=1, n is odd according to the definition of odd. So n is either
odd or even.

\emph{After the base case, first assume, then push with help of the definition}
Assume n is odd or even, then n+2 is also odd or even according to the
definition of odd or even.

QED

\subsection{Tabloo or truth tree}
\subsubsection{When to use it?}
The tabloo method comes to mind if you have to proof things about systems
that use trees, and where the values of the leaves of the tree trincle down
to the root of the tree.

So for the base case the tabloo is method is used to show that the valuation
trinceling down happens for the rules, then induction is used for the entire
proof tree to show that it works for the entire system.

\subsubsection{How to use it?}
In this strategy you create a tree, starting from the axioms which is either
true or false, you go down step by step and say what happens. In most systems
if one of the axioms is false it means the conclusion is false to.


\paragraph{Proof that for $L\to$ of GKC that if one of the premises is not
true (under all valuations) than so is the conclusion}


\subparagraph{Definitions} GKC is gentzen calculy, in which $L\to$ is defined
as follows:

\begin{prooftree}
	\AxiomC{$\Gamma \Rightarrow  A, \Delta$}
	\AxiomC{$\Gamma, B \Rightarrow \Delta$}
	\RightLabel{$ L\to$ }
	\BinaryInfC{$\Gamma, A \to B \Rightarrow \Delta$}
\end{prooftree}

Where $\Gamma$ is a collection of formulas connected by $\wedge$ and $\Delta$
is a collection of formulas connected by $\vee$.
A and B are singular formulas.
I(x) is a function that changes the $,$ into proper connectives. So in case
of the antecedent it will become a $\wedge$ and in the succedent a $\vee$.

\subparagraph{Strategy} Under all valuations, if one of the premises is not
true of $L\to$ then so is the conclusion. This will be proven with help
of the tabloo method.
\subparagraph{Proof}
There are two cases to consider first $V(I(\Gamma \Rightarrow  A, \Delta))=0$
and secondly $V(I(\Gamma, B \Rightarrow  \Delta))=0$

$V(I(\Gamma \Rightarrow  A, \Delta))=0$ implies $V(A)=0$ and for all formulas
where $C\in \Delta$ and $D \in \Gamma$, $V(C) = 0$ and $V(D)=1$. In this case
$V(A \to B) = 1$ and thus $V(I(\Gamma, A \to B \Rightarrow \Delta))=0$.

\emph{So here I say if the $A$ premise is false then the conclusion is false
	because propositional logic, you need the $I$ function to deal with
	commas. And you need the V function to execute a true false valuation
(ergo execute propositional logic)}

$V(I(\Gamma, B \Rightarrow \Delta))=0$ implies $V(B)=1$ and for all formulas
where $C\in \Delta$ and $D \in \Gamma$, $V(C) = 0$ and $V(D)=1$. In this case
$V(A \to B) = 1$ and thus $V(I(\Gamma, A \to B \Rightarrow \Delta))=0$.

\emph{So step by step, say what the valuation for the premis implies for the
	individual elements (if its a set of elements, just do it on a member of
the set, then show it for the conclusion with help of propositional logic}

\section{Notations}
\label{notations}
There is no standart in notating things. Every logician seems to use a
slightly different notations. Keep this in mind. But I think the connectives
are mostly the same. You probably need to memorize these.

\noindent
\begin{tabular}{@{}lll@{}}
Symbols&
In natural language &
Technical name \\ \toprule
$\neg$ 			& not 				& negation \\
$\wedge$,\& 	& and 				& conjunction \\
$\vee$ 			& or 				& disjunction\\
$\to$	& if \ldots then 	& implication\\
$\leftrightarrow$	& if and only if 	& equivalent\\ \bottomrule
\end{tabular}


Then there are some other symbols used:

\noindent
\begin{tabular}{@{}ll@{}}
Symbols&
In natural language \\ \toprule
$\vdash$ 					& Part of system, ie, is derivable \\
$\vdash, \Rightarrow$ 		& Seperator between antecedent and succedent\\ \bottomrule
\end{tabular}


\section{Systems}
\begin{itemize}
	\item $H$  Hilbert, has a huge set of axioms, and only 1 rule. (also called fredge)
	\item $CPC$ Clasicial Propositional logic, formulas in here are always true
(tautologies)
	\item $GKC$ Gentzen calculus, also called sequent calcules
	\item $GKC_K$  Getnzen + R 
	\item $GKP$ Genzen + structural rules.
\end{itemize}

\section{Proving in Hilbert}
Hilbert is the most simple system. However because of its simplicity proving
things in it isn't very obvious. There are multiple ways of writing proovs

\section{Proving in Grammar Logic}
Grammar logic is a (failed) attempt to make sense of natural langauge. It
tries to organize words into types, and groups of those words into compositional
types.
Connection words such as ` that' will have complex types.

\end{document}
